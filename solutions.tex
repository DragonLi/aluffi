\documentclass[14pt,parskip=full]{scrartcl}

\usepackage{amsthm}
\usepackage{amsmath}
\usepackage{amssymb}

\usepackage[T1]{fontenc}
\usepackage[utf8]{inputenc}
\usepackage{pdfrender,xcolor}
\usepackage{mathrsfs}

\usepackage[margin=0.6in]{geometry}
\usepackage{breqn}

\usepackage{chngcntr}
\counterwithout{subsection}{section}

\setlength{\parindent}{30pt}

\setkomafont{disposition}{\normalfont\bfseries}

\newcommand{\exercise}[3]{
\noindent
\textbf{#1.} #2

\noindent
\textit{Solution.}{\let\tpar\par \let\par\relax #3}
}

\newcommand{\set}[1]{\left\{\,#1\,\right\}}
\newcommand*{\QEDA}{\hfill\ensuremath{\blacksquare}}
\newcommand{\id}{\mathrm{id}}
\newcommand{\im}{\mathrm{im}}

% category theory macros
\newcommand{\Obj}{\mathrm{Obj}}
\newcommand{\Hom}{\mathrm{Hom}}

\newcommand\abs[1]{\left|#1\right|}

\renewcommand{\thesection}{\Roman{section}} 

\begin{document}
\pdfrender{StrokeColor=black,TextRenderingMode=2,LineWidth=0.6pt}

\section{Chapter 1}
\subsection{Preliminaries: Set theory and categories}

\exercise
{1.1}{Locate a discussion of Russel's paradox, and understand it.}{

Recall that, in naive set theory, any collection of objects that satisfy some
property can be called a set. Russel's paradox can be illustrated as follows.
Let $R$ be the set of all sets that do not contain themselves. Then, if $R\notin
R$, then by definition it must be the case that $R\in R$; similarly, if $R\in R$
then it must be the case that $R\notin R$.

}

\exercise
{1.2}{$\rhd$ Prove that if $\sim$ is an equivalence relation on a set $S$, then
the corresponding family $\mathscr{P}_{\sim}$ defined in \S1.5 is indeed a
partition of $S$; that is, its elements are nonempty, disjoint, and their union
is $S$.  [\S1.5]}{

Let $S$ be a set with an equivalence relation $\sim$. Consider the family of
equivalence classes w.r.t. $\sim$ over $S$:


\[ \mathscr{P}_{\sim} = \set{[a]_{\sim} \mid a\in S} \]

Let $[a]_{\sim}\in\mathscr{P}_{\sim}$. Since $\sim$ is an equivalence relation,
by reflexivity we have $a\sim a$, so $[a]_{\sim}$ is nonempty. Now, suppose
$a$ and $b$ are arbitrary elements in $S$ such that $a\not\sim b$. For
contradiction, suppose that there is an $x\in [a]_{\sim}\cap[b]_{\sim}$. This
means that $x\sim a$ and $x\sim b$. By transitivity, we get that $a\sim b$; this
is a contradiction. Hence the $[a]_{\sim}$ are disjoint. Finally, let $x\in S$.
Then $x\in[x]_{\sim}\in \mathscr{P}_{\sim}$. This means that

\[ \bigcup_{[a]_{\sim} \in \mathscr{P}_{\sim}} [a]_{\sim} = S, \]

that is, the union of the elements of $\mathscr{P}_{\sim}$ is $S$. \QEDA

}

\exercise
{1.3}{$\rhd$ Given a partition $\mathscr{P}$ on a set $S$, show how to define a
relation $\sim$ such that $\mathscr{P} = \mathscr{P}_{\sim}$. [\S1.5]}{

Define, for $a,b\in S$, $a\sim b$ if and only if there exists an
$X\in\mathscr{P}$ such that $a\in X$ and $b\in X$. We will show that
$\mathscr{P} = \mathscr{P}_{\sim}$. 

\tpar
1. ($\mathscr{P}\subseteq\mathscr{P}_{\sim}$). Let $X\in \mathscr{P}$; we want to
show that $X\in\mathscr{P}_{\sim}$. We know that $X$ is
nonempty, so choose $a\in X$ and consider $[a]_{\sim}\in\mathscr{P}_{\sim}$. We
need to show that $X=[a]_{\sim}$. Suppose $a'\in X$ (it may be that $a'=a$.)
Since $a,a'\in X$, $a\sim a'$, so $a'\in[a]_{\sim}$. Now, suppose $a'\in
[a]_{\sim}$. We have $a'\sim a$, so $a'\in X$. Hence $X=[a]_{\sim}\in
\mathscr{P}_{\sim}$, so $\mathscr{P}\subseteq\mathscr{P}_{\sim}$.

\tpar
2. ($\mathscr{P}_{\sim}\subseteq\mathscr{P}$). Let $[a]_{\sim}\in\mathscr{P}_{\sim}$.
From exercise I.1.1 we know that $[a]_{\sim}$ is non-empty. Suppose
$a'\in[a]_{\sim}$. By definition, since $a'\sim a$, there exists a set $X$ such
that $a,a'\in X$. Hence $[a]_{\sim}\subseteq X$. Also, if $a,a'\in X$ (not
necessarily distinct) then $a\sim a'$. Therefore,
$\mathscr{P}_{\sim}\subseteq\mathscr{P}$, and with 1. we get that the sets
$\mathscr{P}$ and $\mathscr{P}_{\sim}$ are equal.

\QEDA

}

\exercise
{1.4}{How many different equivalence relations can be defined on the set
$\{1,2,3\}?$}{

From the definition of an equivalence relation and the solution to problem
\textbf{I.1.3}, we can see that an equivalence relation on $S$ is equivalent to
a partition of $S$. Thus the number of equivalence relations on $S$ is equal to
the number of partitions of $S$. Since $\{1,2,3\}$ is small we can determine
this by hand:

\tpar
\[ \mathscr{P}_0 = \set{\{1,2,3\}} \]
\[ \mathscr{P}_1 = \set{\{1\},\{2\},\{3\}\}} \]
\[ \mathscr{P}_2 = \set{\{1,2\},\{3\}} \]
\[ \mathscr{P}_3 = \set{\{1\},\{2,3\}} \]
\[ \mathscr{P}_4 = \set{\{1,3\},\{2\}} \]

\tpar
\noindent
Thus there can be only $5$ equivalence relations defined on $\{1,2,3\}$.
\QEDA

}

\exercise
{1.5}{Give an example of a relation that is reflexive and symmetric but not
transitive. What happens if you attempt to use this relation to define a
partition on the set? (Hint: Thinking about the second question will help you
answer the first one.)}{

For $a,b\in \mathbf{Z}$, define $a\diamond b$ to be true if and only if
$\abs{a-b}
\leq 1$. It is reflexive, since $a\diamond a = \abs{a-a} = 0 \leq 1$ for any $a\in
\mathbf{Z}$, and it is symmetric since $a\diamond b = \abs{a-b} = \abs{b-a} = b\diamond
a$ for any $a,b\in \mathbf{Z}$. However, it is not transitive. Take for example
$a=0, b=1, c=2$.  Then we have $\abs{a-b} = 1\leq 1$, and $\abs{b-c} = 1\leq 1$, but
$\abs{a-c} = 2 > 1$; so $a\diamond b$ and $b\diamond c$, but not $a\diamond c$.

\tpar
When we try to build a partition of $\mathbf{Z}$ using $\diamond$, we get
"equivalence classes" that are not disjoint. For example, $[2]_{\diamond} =
\{1,2,3\}$, but $[3]_{\diamond} = \{2,3,4\}$. Hence $\mathscr{P}_{\diamond}$ is
not a partition of $\mathbf{Z}$.
\QEDA

}

\exercise
{1.6}{Define a relation $\sim$ on the set $\mathbf{R}$ of real numbers by
setting $a\sim b\iff b-a\in\mathbf{Z}$. Prove that this is an equivalence
relation, and find a `compelling' description for $\mathbf{R}\,/\!\sim$. Do the
same for the relation $\approx$ on the plane $\mathbf{R}\times\mathbf{R}$ by
declaring $(a_1,b_1)\approx(a_2,b_2)\iff b_1-a_1\in\mathbf{Z}$ and
$b_2-a_2\in\mathbf{Z}$. [\S II.8.1, II.8.10]}{

Suppose $a,b,c\in\mathbf{R}$. We have that $a-a=0\in\mathbf{Z}$, so $\sim$ is
reflexive. If $a\sim b$, then $b-a=k$ for some $k\in\mathbf{Z}$, so
$a-b=-k\in\mathbf{Z}$, hence $b\sim a$. So $\sim$ is symmetric. Now, suppose that
$a\sim b$ and $b\sim c$, in particular that $b-a=k\in\mathbf{Z}$ and
$c-b=l\in\mathbf{Z}$. Then $c-a=(c-b) + (b-a) = l+k\in\mathbf{Z}$, so $a\sim c$.
So $\sim$ is transitive.

\tpar
An equivalence class $[a]_{\sim}\in\mathbf{R}\,/\!\sim$ is the set of integers
$\mathbf{Z}$ transposed by some real number $\epsilon\in[0,1)$. That is, for
every set $X\in\mathbf{R}\,/\!\sim$, there is a real number $\epsilon\in[0,1)$
such that every $x\in X$ is of the form $k+\epsilon$ for some integer $k$.

\tpar
Now we will show that $\approx$ is an equivalence relation over
$\mathbf{R}\times\mathbf{R}$. Supposing $a_1,a_2\in\mathbf{R}\times\mathbf{R}$,
we have $a_1-a_1=a_2-a_2=0\in\mathbf{Z}$, so $(a_1,a_2)\approx(a_1,a_2)$. If we
also suppose that $b_1,b_2,c_1,c_2\in\mathbf{R}\times\mathbf{R}$, then symmetry
and transitivity can be shown as well: $(a_1,a_2)\approx(b_1,b_2)\implies
b_1-a_1=k$ for some integer $k$ and $b_2-a_2=l$ for some integer $l$, hence
$a_1-b_1=-k\in\mathbf{Z}$ and $a_2-b_2=-l\in\mathbf{Z}$, so
$(b_1,b_2)\approx(a_1,a_2)$; also if $(a_1,a_2)\approx(b_1,b_2)$ and
$(b_1,b_2)\approx(c_1,c_2)$, then
$(b_1,b_2)-(a_1,a_2)=(k_1,k_2)\in\mathbf{Z}\times\mathbf{Z}$ as well as
$(c_1,c_2)-(b_1,b_2)=(l_1,l_2)\in\mathbf{Z}\times\mathbf{Z}$, so $(c_1,c_2) -
(a_1,a_2) = (c_1,c_2) - (b_1,b_2) + (b_1,b_2) - (a_1,a_2) = (k_1+l_1,
k_2+l_2\in\mathbf{Z}\times\mathbf{Z}$. Thus $\approx$ is an equivalence
relation.

\tpar
The interpretation of $\approx$ is similar to $\sim$. An equivalence class
$X\in\mathbf{R}\times\mathbf{R}\,/\approx$ is just the 2-dimensional integer
lattice $\mathbf{Z}\times\mathbf{Z}$ transposed by some pair of values
$(\epsilon_1,\epsilon_2)\in[0,1)\times[0,1)$.
\QEDA

}

\subsection{Functions between sets}

\exercise
{2.1}{How many different bijections are there between a set with $n$ elements
and itself?}{

A function $f:S\to S$ is a graph $\Gamma_f\subseteq S\times S$. Since $f$ is
bijective, then for all $y\in S$ there exists a unique $x\in S$ such that
$(x,y)\in\Gamma_f$. We can see that $\abs{\Gamma_f} = n$. Since each $x$ must be
unique, all the elements $x\in S$ must be present in the first component of
exactly one pair in $\Gamma_f$. Furthermore, if we order the elements $(x,y)$ in
$\Gamma_f$ by the first component, we can see that $\Gamma_f$ is just a
permutation on the $n$ elements in $S$. For example, for $S=\{1,2,3\}$ one such
$\Gamma_f$ is:

\[ \set{ (1,3), (2,2), (3,1) } \]

Since $\abs{S} = n$, the number of permutations of $S$ is $n!$. Hence there can be
$n!$ different bijections between $S$ and itself.
\QEDA

}

\exercise
{2.2}{$\rhd$ Prove statement (2) in Proposition 2.1. You may assume that given a
family of disjoint subsets of a set, there is a way to choose one element in
each member of the family. [\S2.5, V3.3]

\begin{quote} \textbf{Proposition 2.1.} Assume $A\neq\empty$, and let $f:A\to B$ be a function. Then
(1) $f$ has a left-inverse if and only if $f$ is injective; and \\
(2) $f$ has a right-inverse if and only if $f$ is surjective.
\end{quote}}{

Let $A\neq\empty$ and suppose $f:A\to B$ is a function.

\tpar
($\implies$) Suppose there exists a function $g$ that is a right-inverse of $f$.
Then $f\circ g = \id_A$. Let $b\in B$. We have that $f(g(b)) = b$, so there
exists an $a = g(b)$ such that $f(a) = b$. Hence $f$ is surjective.

\tpar
($\impliedby$) Suppose that $f$ is surjective. We want to construct a function
$g:B\to A$ such that $f(g(a)) = a$ for all $a\in A$. Since $f$ is surjective,
for all $b\in B$ there is an $a\in A$ such that $f(a) = b$. For each $b\in B$
construct a set $\Lambda_b$ of such pairs:

\[ \Lambda_b = \set{ (a,b) \mid a \in A, f(a) = b } \]

Note that $\Lambda_b$ is non-empty for all $b\in B$. So that we can choose one
pair $(a,b)$ ($a$ not necessarily unique) from each set in $\Lambda =
\set{\Lambda_b\mid b\in B}$ to define $g:B\to A$:

\[ g(b) = a, \text{ where $a$ is in some $(a,b)\in\Lambda_b$} \]

Now, $g$ is a right-inverse of $f$. To show this, let $b\in B$. Since $f$ in
surjective, $g$ has been defined such that when $a=g(b)$, $f(a)=b$, so we get
that $f(g(b)) = (f\circ g)(b) = b$, thus $g$ is a right-inverse of $f$.
\QEDA

}

\exercise
{2.3}{Prove that the inverse of a bijection is a bijection and that the
composition of two bijections is a bijection.}{

(1) Suppose $f:A\to B$ is a bijection, and that $f^{-1}:B\to A$ is its inverse.
We have that $f\circ f^{-1} = \id_B$ and $f^{-1}\circ f = \id_A$. Hence $f$ is
the left- and right-inverse of $f^{-1}$, so $f^{-1}$ must be a bijection.
\QEDA

\tpar
(2) Let $f:B\to C$ and $g:A\to B$ be bijections, and consider $f\circ g$. To
show that $f$ is injective, let $a, a'\in A$ such that $(f\circ g)(a) = (f\circ
g)(a')$. Since $f$ is a bijection, $f(g(a)) = f(g(a')) \implies g(a) = g(a')$.
Also, since $g$ is a bijection, $g(a) = g(a') \implies a=a'$. Hence $f\circ g$
is injective. Now, let $c\in C$. Since $f$ is surjective, there is a $b\in B$
such that $f(b) = c$. Also, since $g$ is surjective, there is an $a\in A$ such
that $g(a) = b$; this means that there is an $a\in A$ such that $(f\circ g)(a) =
c$. So $f\circ g$ is bijective.

}

\exercise{2.4}{$\rhd$ Prove that `isomorphism' is an equivalence relation (on any set
of sets.) [\S4.1]}{

Let $S$ be a set. Then $\id_S$ is a bijection from $S$ to itself, so $S\cong S$.
Let $T$ be another set with $S\cong T$, i.e. that there exists a bijection
$f:S\to T$. Since $f$ is a bijection, it has an inverse $f^{-1}:T\to S$, so
$T\cong S$. Finally, let $U$ also be a set, and assume that there exists
bijections $f:S\to T$ and $g:T\to U$, i.e. that $S\cong T$ and $T\cong U$. From
exercise \textbf{I.2.3} we know that the composition of bijections is itself a
bijection. This means that $g\circ f: S\to U$ is a bijection, so $S\cong U$.
Hence $\cong$ is an equivalence relation.
\QEDA

}

\exercise{2.5}{$\rhd$ Formulate a notion of \textit{epimorphism}, in the style
of the notion of \textit{monomorphism} seen in \S 2.6, and prove a result
analogous to Proposition 2.3, for epimorphisms and surjections.}{
A function $f:A\to B$ is an \textit{epimorphism} if and only if for all sets $Z$
and all functions $b':Z\to B$, there is a function $a':Z\to A$ such that $f\circ
a' = b'$. Now we will show that $f$ is a surjection if and only if it is an
epimorphism.

\tpar
($\implies$) Suppose that $f:A\to B$ is surjective. Let $Z$ be a set and
$b':Z\to B$ a function. We need to construct a function $a':Z\to A$ such that
$f\circ a' = b'$. Fix $z\in Z$. Suppose $b=b'(z)\in B$. Since $b\in B$ and $f$
is surjective, there exists an $a\in A$ such $f(a) = b$. Now, define $a'(z) =
a$ this way for each $z\in Z$. Then $f\circ a'(z) = b'(z)$ for all $z\in Z$, so
$f\circ a' = b'$. Hence $f$ is an epimorphism.

\tpar
($\impliedby$) Suppose that $f$ is an epimorphism. Let $b':B\to B$ be a
bijection. Since $f$ is an epimophism, there is a function $a':B\to A$ such that
$f\circ a' = b'$. Let $b\in B$. Since $b'$ is a bijection, there is a unique
element $y\in B$ such that $b'(y) = b$. Furthermore, we have that $(f\circ
a')(y) = b$. In other words, $a = a'(y)$ is an element in $a$ such that $f(a) =
b$. Hence $f$ is surjective, as required.
\QEDA

}

\exercise{2.6}{With notation as in Example 2.4, explain how any function $f:A\to
B$ determines a section of $\pi_A$.}{

Let $f:A\to B$ and let $\pi_A:A\times B\to A$ be such that $\pi_A(a,b) = a$ for
all $(a,b)\in A\times B$. Construct $g:A\to A\times B$ defined as $g(a) = (a,
f(a))$ for all $a\in A$. The function $g$ can be thought of as `determined by'
$f$. Now, since $(\pi_A\circ g)(a) = \pi_A(g(a)) = \pi_A(a, f(a)) = a$ for all
$a\in A$, $g$ is a right inverse of $\pi_A$, i.e. $g$ is a section of $\pi_A$ as
required.
\QEDA

}

\exercise{2.7}{Let $f:A\to B$ be any function. Prove that the graph $\Gamma_f$
of $f$ is isomorphic to $A$.}{

Recall that sets $\Gamma_A$ and $A$ are \textit{isomorphic}, written
$\Gamma_A\cong A$, if and only if there exists a bijection $g:\Gamma_A\to A$.
Let's construct such a function $g$, defined to be $g(a,b) = a$. Keep in mind
that here $(a,b)\in\Gamma_f\subseteq A\times B$.

\tpar
Let $(a',b'),(a'',b'')\in\Gamma_f$ such that $f(a',b') = f(a'',b'')$. For
contradiction, suppose that $(a',b')\neq (a'',b'')$. Since $f(a',b') = a' = a''
= f(a'',b'')$, it must be that $b'\neq b''$. However, this would mean that both
$(a',b')$ and $(a',b'')$ are in $\Gamma_f$; this would mean that $f(a') = b'
\neq b'' = f(a')$, which is impossible since $f$ is a function. Hence $g$ is
injective.

\tpar
Let $a'\in A$. Since $f$ is a well-defined function with $A$ as its domain,
there must exists a pair $(a',b')\in\Gamma_f$ for some $b'\in B$, in particular
that $g(a',b') = a'$; thus $g$ is surjective, so it is a bijection.
\QEDA

}

\exercise{2.8}{Describe as explicitly as you can all terms in the canonical
decomposition (cf. \S2.8) of the function $\mathbf{R}\to\mathbf{C}$ defined by
$r\mapsto e^{2\pi ir}$. (This exercise matches one previously. Which one?)}{

Let $f:\mathbf{R}\to\mathbf{C}$ be as above. The first piece in the canonical
decomposition is the equivalence relation $\sim$ defined as $x \sim x' \iff f(x) =
f(x')$, i.e. $[x]_{\sim}$ is the set of all elements in $\mathbf{R}$ that get
mapped to the same element in $\mathbf{C}$ by $f$ as $x$.

\tpar
The second piece is the set $\mathscr{P}_{\sim}$. This set is the set of all
equivalence classes of $\mathbf{R}$ over equality up to $f$. Note that, since
$f(x) = e^{2\pi i x} = \cos(2\pi x) + i\sin(2\pi x)$, $f$ is periodic with
period $1$. That is, $f(x) = e^{2\pi i x} = e^{2\pi i x + 2\pi} = e^{2\pi i (x +
1)} = f(x+1)$. In other words, we can write $\mathscr{P}_{\sim}$ as,

\[ \mathscr{P}_{\sim} = \set{\set{r + k\mid k\in\mathbf{Z}}\mid
r\in[0,1)\subseteq\mathbf{R}}, \]

and it is here when we notice uncanny similarities to exercise \textbf{1.6}
where $x\sim y$, for $x,y\in\mathbf{R}$, if and only if $x-y\in\mathbf{Z}$, in
which we could have written $\mathscr{P}_{\sim}$ in the same way. 

\tpar
Now we will explain the mysterious $\tilde{f}:\mathscr{P}_{\sim}\to\im f$. This
function is taking each \textit{equivalence class} $[x]_{\sim}$ over the reals
w.r.t. $\sim$ and mapping it to the element in $\mathbf{C}$ that $f$ maps each
element $x'\in[x]_{\sim}$ to; indeed, since $x\sim x'$ is true for
$x,x'\in\mathbf{R}$ if and only if $f(x)=f(x')$, we can see that for any
$x\in\mathbf{R}$, for all $x'\in[x]_{\sim}$, there exists a $c\in\mathbf{C}$
such that $f(x') = c$. To illustrate with the equivalence class over
$\mathbf{R}$ w.r.t. $\sim$ corresponding to the element $0\in\mathbf{R}$, we
have $[0]_{\sim} = \set{\dots, -2, -1, 0, 1, 2, \dots}$.  We can see that
$e^{-4\pi i} = e^{-2\pi i} = e^{0\pi i} = 1 = e^{2\pi i} = e^{4\pi i}$, etc; so
the function would map $[0]_{\sim}\mapsto1\in\mathbf{C}$, and so on.
Furthermore, we can see that $\tilde{f}$ is surjective, since for $y$ to be in
$\im f$ is to say that there is an $x\in\mathbf{R}$ such that $f(x) = y$; so
there must be an equivalence class $[x]_{\sim}$ which is mapped to $y$ by
$\tilde{f}$.

\tpar
Finally, the simple map from $\im f\to\mathbf{C}$ that simply takes $c\mapsto
c$. This can be thought of as a potential ``expansion'' of the domain of
$\tilde{f}$. It is obviously injective, since (trivially) $c\neq c'\implies
c\neq c'$. However, it may not be surjective: for example, $2\in\mathbf{C}$ is
not in $\im f$ as it is defined above.

}

\exercise{2.9}{$\rhd$ Show that if $A'\cong A''$ and $B'\cong B''$, and further
$A'\cap B'=\emptyset$ and $A''\cap B''=\emptyset$, then $A'\cup B'\cong A''\cup
B''$. Conclude that the operation $A\sqcup B$ is well-defined up to
\textit{isomorphism} (cf. \S2.9) [\S2.9, 5.7]}{

Let $A',A'',B',B''$ be sets as described above. Since $A'\cong A''$ and $B'\cong
B''$, we know there exists respective bijections $f:A'\to A''$ and $g:B'\to
B''$. Now, we wish to show that $A'\cup B'\cong A''\cup B''$. Define a function
$h:A'\cup B'\to A''\cup B''$ such that $h(x) = f(x)$ if $x\in A'$ and $g(x)$ if
$x\in B'$.

\tpar
We will now show that $h$ is a bijection. Let $y\in A''\cup B''$. Then, since
$A''\cap B''=\emptyset$, either $y\in A''$ or $y\in B''$. Without loss of
generality suppose that $y\in A''$. Then, since $f:A'\to A''$ is a bijection, it
is \textit{surjective}, so there exists an $x\in A'\subseteq A'\cup B'$ such
that $h(x) = f(x) = y$. So $h$ is surjective. Now, suppose that $x\neq x'$, for
$x,x'\in A'\cup B'$. If $x,x'\in A'$, then since $f$ is injective and $h(x) =
f(x)$ for all $x\in A'$, then $h(x)\neq h(x')$. Similarly for if $x,x'\in B'$.
Now, without loss of generality if $x\in A'$ and $x'\in B'$, then $h(x) = f(x)
\neq g(x') = h(x')$ since $A''\cap B''=\emptyset$. Hence $h$ is a bijection, so
$A'\cup B'\cong A''\cup B''$.

\tpar
Since these constructions of $A',A'',B',B''$ correspond to creating ``copies''
of sets $A$ and $B$ for use in the disjoint union operation, we have that
disjoint union is a well-defined function \textit{up to isomorphism}. In
particular, since $\cong$ is an equivalence relation, we can consider $\sqcup$
to be well-defined from $\mathscr{P}_{\cong}$ to $A'\cup B'$.
\QEDA

}

\exercise
{2.10}{$\rhd$ Show that if $A$ and $B$ are finite sets, then $\abs{B^A} =
\abs{B}^{\abs{A}}$. [\S2.1, 2.11, I.4.1]}{

Let $A$ and $B$ be sets with $\abs{A}=n$ and $\abs{B}=m$, with $n,m$ being
non-negative integers. Recall that $B^A$ denotes the set of functions $f:A\to
B$. Now, if $A=B=\emptyset$ or $A=\emptyset$ and $\abs{B}=1$, we get one
function, the empty function $\Gamma_f = \emptyset$, and $0^0 = 1^0 = 1$. If
$\abs{A} = \abs{B} = 1$, then we get the singleton function
$\Gamma_f=\{(a,b)\}$, and $1^1 = 1$. If $A\neq\emptyset$ and $B=\emptyset$, then
no well-defined function can exist from $A$ to $B$ since there will be no value
for the elements in $A$ to take; this explains $\abs{B^A} = \abs{B}^{\abs{A}} =
0^{\abs{A}} = 0$.

\tpar
Suppose that $B\neq\emptyset$ and $B$ is finite. We will show inductively that
$\abs{B^A} = \abs{B}^{\abs{A}}$. First, suppose that $\abs{A} = 1$.  Then there
are exactly $\abs{B}$ functions from $A$ to $B$: if $B=\set{b_1,b_2,\dots,b_m}$,
then the functions are $\{(a,b_1)\}, \{(a,b_2)\}$, etc. Hence $\abs{B^A} =
\abs{B}^{\abs{A}} = \abs{B}$.  Now, fix $k\geq 2$, and assume that $\abs{B^A} =
\abs{B}^{\abs{A}}$ for all sets $A$ such that $\abs{A}=k-1$. Suppose that
$\abs{A}=k$. Let $a\in A$. (We can do this since $\abs{A}=k\geq 2$.) Then, by
the inductive hypothesis, since $\abs{A\backslash\{a\}}=k-1$,
$\abs{B^{(A\backslash\{a\})}} = \abs{B}^{\abs{A}-1}$. Let $F$ be the set of
functions from $A\backslash\{a\}$ to $B$.  Then, for each of those functions
$f\in F$, there is $\abs{B}$ ``choices'' of where to assign $a$: one choice for
each element in $B$. Hence, $\abs{B^A} = \abs{B}\abs{B}^{\abs{A}-1} =
\abs{B}^{\abs{A}}$ as required.
\QEDA

}

\exercise
{2.11}{$\rhd$ In view of Exercise 2.10, it is not unreasonable to use $2^A$ to denote
the set of functions from an arbitrary set $A$ to a set with $2$ elements (say
$\{0,1\}$). Prove that there is a bijection between $2^A$ and the \textit{power
set} of $A$ (cf. \S1.2). [\S1.2, III.2.3]}{

Let $S = \{0,1\}$, and consider $f:\mathcal{P}(A)\to 2^A$, defined as

\[ f(X) = \set{(a,1) \text{ if $a\in X$, and }(a,0) \text{ otherwise}} \]

We will show that $f$ is bijective. Let $g\in 2^A$. Then $f$ is a
function from $A$ to $S$. Let $A_1 = \set{a\in A\mid g(a) = 1}$. Then $A_1$ is a
set such that $A_1\in\mathcal{P}(A)$, and $f(A_1)=g$. Hence $f$ is surjective.

Now, suppose that $X,Y\subseteq A$ and $f(X) = f(Y)$. Then, for all $a\in A$,
$a\in X \iff f(X)(a) = 1 \iff f(Y)(a) = 1 \iff a\in Y$. Hence $f$ is injective,
so $2^A\cong\mathcal{P}(A)$.
\QEDA

}

\subsection{Category theory}

\exercise
{3.1}{$\rhd$ Let $\mathsf{C}$ be a category. Consider a structure
$\mathsf{C}^{op}$ with
\begin{enumerate}
\item $\Obj(\mathsf{C}^{op}) = \Obj(\mathsf{C})$
\item For $A, B$ objects of $\mathsf{C}^{op}$ (hence objects of $\mathsf{C}$),
$\Hom_{\mathsf{C}^{op}}(A, B) := \Hom_{\mathsf{C}}(B, A)$.
\end{enumerate}
Show how to make this into a category (that is, define composition of morphisms
in $\mathsf{C}^{op}$ and verify the properties listed in \S3.1).

Intuitively, the `opposite' category $\mathsf{C}^{op}$ is simply obtained by
`reversing all the arrows' in $\mathsf{C}$. [5.1, \S III.1.1, \S IX.1.2,
IX.1.10]}{

For objects $A,B,C\in\Obj(\mathsf{C}^{op})$, the set of morphisms between $A$
and $B$ in $\mathsf{C}^{op}$, $\Hom_{\mathsf{C}^{op}}(A,B)$, is defined as
$\Hom_{\mathsf{C}}(B, A)$. Similarly for the morphisms between $B$ and $C$. So
for morphisms $f\in\Hom_{\mathsf{C}^{op}}(A,B)$ and
$g\in\Hom_{\mathsf{C}^{op}}(B,C)$, to define composition we recall the
set-function $\circ_{\mathsf{C}} : \Hom_{\mathsf{C}}(C,B) \times
\Hom_{\mathsf{C}}(B,A) \to\Hom_{\mathsf{C}}(C,A)$ that is defined for the
objects $A,B,C\in\Obj(\mathsf{C})=\Obj(\mathsf{C}^{op})$; we shall define the
composition of morphisms $f:A\to B$ and $g:B\to C$ in $\mathsf{C}^{op}$ with
this function. Precisely, we define

\[ \circ_{\mathsf{C}^{op}}: \Hom_{\mathsf{C}^{op}}(A,B) \times
\Hom_{\mathsf{C}^{op}}(B,C) \to \Hom_{\mathsf{C}^{op}}(A,C) \]

to be

\[ \circ_{\mathsf{C}^{op}}(f,g) = \circ_{\mathsf{C}}(g,f) \]

for all $f\in\Hom_{\mathsf{C}^{op}}(A,B)$ and $f\in\Hom_{\mathsf{C}^{op}}(B,C)$.
The domain and codomain of $\circ_{\mathsf{C}}$ and $\circ_{\mathsf{C}^{op}}$
match (up to transposing the coordinates in the domain) due to the equality of
$\Hom_{\mathsf{C}}(A,B)$ with $\Hom_{\mathsf{C}^{op}}(B,A)$.

\tpar
To show that this composition makes $\mathsf{C}^{op}$ a category, first we note
that the fact that $\mathsf{C}$ is a category implies the existence of a
morphism $1_{A}$ taking $A$ to itself where $A\in\Obj(\mathsf{C})$; this
morphism is thus also present in $\Hom_{\mathsf{C}^{op}}(A,A) =
\Hom_{\mathsf{C}}(A,A)$. Secondly, for objects $A,B,C,D\in\Obj(\mathsf{C})$, any
morphisms $f\in\Hom_{\mathsf{C}^{op}}(A,B)$, $g\in\Hom_{\mathsf{C}^{op}}(B,C)$,
and $h\in\Hom_{\mathsf{C}^{op}}(C,D)$ are associative, since

\[ (h \circ_{\mathsf{C}^{op}} g) \circ_{\mathsf{C}^{op}} f = f
\circ_{\mathsf{C}} (g \circ_{\mathsf{C}} h) = (f \circ_{\mathsf{C}} g)
\circ_{\mathsf{C}} h = h \circ_{\mathsf{C}^{op}} (g \circ_{\mathsf{C}^{op}} f).
\]

Finally, for any morphism $f\in\Hom_{\mathsf{C}^{op}}(A,B)$ we have,

\[ f \circ_{\mathsf{C}^{op}} 1_A = 1_A \circ_{\mathsf{C}} f = f \text{ and }
1_B \circ_{\mathsf{C}^{op}} f = f \circ_{\mathsf{C}} 1_B = f; \]

hence the identities are ``identities with respect to composition''. Last, for
objects $A,B,C,D\in\Obj(\mathsf{C})$ where $A\neq C$ and $B\neq D$,
clearly $\Hom_{\mathsf{C}}(B,A)\cap\Hom_{\mathsf{C}}(D,C)=\emptyset$ is true iff 
$\Hom_{\mathsf{C^{op}}}(A,B)\cap\Hom_{\mathsf{C^{op}}}(C,D)=\emptyset$. Hence
$\mathsf{C}^{op}$ is a category.
\QEDA

}

\exercise
{3.2}{If $A$ is a finite set, how large is $\mathrm{End}_{\mathsf{Set}}(A)$?}{

The set $\mathrm{End}_{\mathsf{Set}}(A)$ is the set of functions $f:A\to A$.
Since $A$ is finite, write $\abs{A} = n$ for some $n\in\mathbf{Z}$. By exercise
2.10, we know that $\abs{A^A} = \abs{A}^{\abs{A}} = n^n$. So the the set
$\mathrm{End}_{\mathsf{Set}}(A)$ has size $n^n$.
\QEDA

}

\exercise
{3.3}{$\rhd$ Formulate precisely what it means to say that $1_a$ is an identity
with respect to composition in Example 3.3, and prove this assertion. [\S3.2]}{

Let $S$ be a set and $\sim$ be a binary relation on the set $S$. Then, for
elements $a,b\in S$, $\Hom(a,b)$ is the pair $(a,b)\in S\times S$ if $a\sim b$,
or $\emptyset$ otherwise. Composition of morphisms $(a,b)$ and $(b,c)$ is simply
the pair $(a,c)$, which captures the transitivity of $\sim$. We will say that
$1_a = (a,a)$, for $a\in S$, is \textit{an identity with respect to composition}
if, for any $b\in S$, $(a,b)(a,a) = (a,b)$. Now, if $a\sim a$ and $a\sim b$,
then trivially it is the case that $a\sim b$; hence $(a,b)(a,a) = (a,b)$, and
$1_a$ is an identity w.r.t. composition as required.
\QEDA

}

\exercise
{3.4}{Can we define a category in the style of Example 3.3 using the relation
$<$ on the set $\mathbf{Z}$?}{

No, we can't. This is because $<$ isn't reflexive: $x\not<x$ for any
$x\in\mathbf{Z}$.
\QEDA

}

\exercise
{3.5}{$\rhd$ Explain in what sense Example 3.4 is an instance of the categories
considered in Example 3.3. [\S 3.2]}{

Let $S$ be a set. Example 3.4 considers the category $\hat{S}$ with objects
$\Obj(\hat{S}) = \mathscr{P}(S)$ and morphisms $\Hom_{\hat{S}}(A,B) =
\set{(A,B)}$ if $A\subseteq B$ and $\emptyset$ otherwise, for all sets
$A,B\in\mathscr{P}$. The category $\hat{S}$ is an instance of the categories
explained in Example 3.3 because $\subseteq$ is a reflexive and transitive
operation on the power set of any set $S$. Indeed, for $X,Y,Z\subseteq S$, we have
that $X\subseteq X$ and, if $X\subseteq Y$ and $Y\subseteq Z$, then if $x\in X$,
then $x\in Y$ and $x\in Z$ so $X\subseteq Z$.
\QEDA

}

\exercise
{3.6}{$\rhd$ (Assuming some familiarity with linear algebra.) Define a category
$\mathsf{V}$ by taking $\Obj(\mathsf{V}) = \mathbf{N}$ and letting
$\Hom_{\mathsf{V}}(m,n) = \text{ the set of $m\times n$ matrices with real
entries,}$ for all $m,n\in\mathbf{N}$. (We will leave the reader the task of
making sense of a matrix with 0 rows or columns.) Use product of matrices to
define composition. Does this category `feel' familiar? [\S VI.2.1, \S
VIII.1.3]}{

Yes! It is yet another instance of Example 3.3. The binary relation $\sim$ on
$\mathbf{N} \times \mathbf{N}$ holds for all values $(n,m)\in\mathbf{N} \times
\mathbf{N}$, and means that a matrix of size $m\times n$ ``can be built''. It is
reflexive trivially. It is transitive trivially as well---a matrix of any size
can be built. However, it would also hold, for example, if we had to in some
sense ``deduce'' that a $3\times 3$ matrix could be built using the fact that
$3\times 1$ and $1\times 3$ matrices can be built.

}

\exercise
{3.7}{$\rhd$ Define carefully the objects and morphisms in Example 3.7, and draw
the diagram corresponding to compositon. [\S 3.2]}{

\def \C {\mathsf{C}}
\def \CA {\mathsf{C}^A}

Let $\C$ be a category, and $A\in\C$. We want to define $\CA$. Let $\Obj(\CA)$
include all morphisms $f\in\Hom_\C(A,Z)$ for all $Z\in\Obj(\C)$. For any two
objects $f,g\in\Obj(\CA)$, $f:A\to Z_1$ and $g:A\to Z_2$, we define the
morphisms $\Hom_\CA(f,g)$ to be the morphisms $\sigma\in\Hom_\C(Z_1, Z_2)$ such
that $g=\sigma f$. Now we must check that these morphisms satisfy the axioms.

\tpar
\begin{enumerate}
  \item Let $f\in\Obj(\CA)\in\Hom_\C(A,Z)$ for some object $Z\in\Obj(\C)$. Then
there exists an identity morphism $1_Z\in\Hom_\C(Z,Z)$ since $\C$ is a category.
This is a morphism such that $f=1_zf$, so $\Hom_\CA(f,f)$ is also nonempty.
  \item Let $f,g,h\in\Obj(\CA)$ such that there are morphisms
$\sigma\in\Hom_\CA(f,g)$ and $\tau\in\Hom_\CA(g,h)$. Then there is a morphism
$\upsilon\in\Hom_\CA(f,h)$, namely $\tau\sigma$, which exists because of
morphism composition in $\C$. For clarity, we write that  $f:A\to Z_1$, $g:A\to
Z_2$, $h:A\to Z_3$, with $\sigma:Z_1\to Z_2$ and $\tau:Z_2\to Z_3$. We have
$g=\sigma f$ and $h=\tau g$. Hence, $\upsilon f = \tau\sigma f = \tau g = h$ as
required.
  \item Lastly, let $f,g,h,i\in\Obj(\CA)$ with $Z_1, Z_2, Z_3, Z_4$ codomains
respectively, and with $\sigma\in\Hom_\CA(f,g)$, $\tau\in\Hom_\CA(g,h)$, and
$\upsilon\in\Hom_\CA(h,i)$. Since $\sigma$, $\tau$, and $\upsilon$ are morphisms
in $\C$ taking $Z_1\to Z_2$, etc., morphism composition is associative; hence
morphism composition is associative in $\CA$ as well.
\end{enumerate}

\QEDA

}

{ \def \C {\mathsf{C}} \def \Cp {\C'}
\exercise
{3.8}{$\rhd$ A \textit{subcategory} $\Cp$ of a category $\C$ consists of a
collection of objects of $\C$, with morphisms
$\Hom_\Cp(A,B) \subseteq \Hom_\C(A,B)$ for all objects $A,B\in\Obj(\Cp)$, such
that identities and compositions in $\C$ make $\Cp$ into a category. A
subcategory $\Cp$ is \textit{full} if $\Hom_\Cp(A,B) = \Hom_\C(A,B)$ for all
$A,B\in\Obj(\Cp)$. Construct a category of \textit{infinite sets} and explain
how it may be viewed as a full subcategory of $\mathsf{Set}$. [4.4,\S VI.1.1, \S
VIII.1.3]}{

\def \Set {\mathsf{Set}}
\def \ISet {\mathsf{Inf}\mathsf{Set}}

Let $\ISet$ be a subcategory of $\Set$ with $\Obj(\ISet)$ being all infinite
sets and $\Hom_\ISet(A,B)$ for infinite sets $A,B$ being the functions from $A$
to $B$. Since $\Hom_\Set(A,B)$ is just the set of all functions from $A$ to $B$
and not, say, the set of all functions from subsets of $A$ that are in
$\Obj(\Set)$ to $B$, $\ISet$ is full since $\Hom_\ISet(A,B)=\Hom_\Set(A,B)$ for
all infinite sets $A,B\in\Obj(\ISet)$.

}}

{\def \Set {\mathsf{Set}} \def \MSet {\mathsf{MSet}}
\exercise
{3.9}{$\rhd$ An alternative to the notion of \textit{multiset} introduced in
\S2.2 is obtained by considering sets endowed with equivalence relations;
equivalent elements are taken to be multiple instance of elements `of the same
kind'. Define a notion of morphism between such enhanced sets, obtaining a
category $\MSet$ containing (a `copy' of) $\Set$ as a full subcategory. (There
may be more than one reasonable way to do this! This is intentionally an
open-ended exercise.) Which objects in $\MSet$ determine ordinary multisets as
defined in \S2.2 and how? Spell out what a morphism of multisets would be from
this point of view. (There are several natural motions of morphisms of
multisets. Try to define morphisms in $\MSet$ so that the notion you obtain for
ordinary multisets captures your intuitive understanding of these objects.)
[\S2.2, \S3.2, 4.5]}{

Define $\Obj(\MSet)$ as all tuples $(S, \sim)$ where $S$ is a set and $\sim$ is
an equivalence relation on $S$. For two multisets $\hat{S} = (S,\sim), \hat{T} =
(T,\approx) \in \Obj(\MSet)$, we define a morphism
$f\in\Hom_\MSet(\hat{S},\hat{T})$ to be a set-function $f:S\to T$ such that, for
$x,y\in S$, $x\sim y\implies f(x)\approx f(y)$, and morphism composition the
same way as set-functions. Now we verify the axioms:

\tpar
\begin{enumerate}
\item For a multiset $(S,\sim)$, we borrow the set-function $1_S:S\to S$ and
note that it necessarily preserves equivalence, i.e. $x\sim y\implies 1_S(x)\sim
1_S(y)$.
\item Let there be objects $\hat{S}=(S,\sim), \hat{T}=(T,\approx),
\hat{U}=(U,\cong)$ with morphisms $f\in\Hom_\MSet(\hat{S},\hat{T})$ and
$g\in\Hom_\MSet(\hat{T},\hat{U})$. Note that $gf:S\to U$ is a set-function since
$\Set$ is a category. Now, since $f$ is a morphism in $\MSet$, for
$x,y\in S$, if $x\sim y$, then $f(x)\approx f(y)$, and since $f(x),f(y)\in T$
and $g$ is a morphism in $\MSet$, $g(f(x))\cong g(f(y))$.
\item Associativity can be proven similarly.
\end{enumerate}

Hence $\MSet$ as defined above is a category. Now, recall that multisets are
defined in \S2.2 as a set $S$ and a \textit{multiplicity function}
$m:S\to\mathbf{N}$. So, for any set $S$ and function $m:S\to\mathbf{N}$, if we
define the equivalence relation corresponding to $m$ as $\sim_m$ then the
tuple $(S,\sim_m)\in\Obj(\MSet)$. The objects in $\MSet$ which
\textit{don't} correspond to any multiset as defined in \S2.2 are sets $S$ with
equivalence relations $\sim$ such that both $S$ and $\mathscr{P}_\sim$ are
uncountable; this way, one cannot construct a function $m:S\to\mathbf{N}$
corresponding to each set in the partition $\mathscr{P}_\sim$, since
$\mathbf{N}$ is countable.

}}

{\def \C {\mathsf{C}} \def \Set {\mathsf{Set}}
\exercise
{3.10}{Since the objects of a category $\C$ are not (necessarily) sets, it is
not clear how to make sense of a notion of `subobject' in general. In some
situations it \textit{does} make sense to talk about subobjects, and the
subobjects of any given object $A$ in $\C$ are in one-to-one correspondence with
the morphisms $A\to\Omega$ for a fixed, special object $\Omega$ of $\C$, called
a \textit{subobject classifier}. Show that $\Set$ has a subobject classifier.}{



}}

{\def \C {\mathsf{C}}
\exercise
{3.11}{$\rhd$ Draw the relevant diagrams and define composition and identities
for the category $\C^{A,B}$ mentioned in Example 3.9. Do the same for the
category $\C^{\alpha,\beta}$ mentioned in Example 3.10. [\S5.5, 5.12]}{

\textit{I will do this with pen and paper. The diagrams are too annoying to
\LaTeX}.

}}

\end{document}
