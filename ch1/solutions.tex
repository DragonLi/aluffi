
% Solutions to exercises in Paolo Aluffi's "Algebra: Chapter 0"
% All solutions copyright 2015 Shane Creighton-Young

% The scrartcl document class lets us have a wider stroke width for the fonts. However, it
% changes the heading font by default so we've used the \setkomafont command to
% restore the headings to the vanilla LaTeX style.
\documentclass[fontsize=14pt]{scrartcl}
\setkomafont{disposition}{\normalfont\bfseries}

% Standard American Math Society packages provide theorem environments, symbols,
% etc.
\usepackage{amsthm}
\usepackage{amsmath}
\usepackage{amssymb}

% Add lcm operator.
\DeclareMathOperator{\lcm}{lcm}

% The tiks-cd package provides macros for easily writing commutative diagrams.
% For more info see [1].
% [1]: http://ctan.math.ca/tex-archive/graphics/pgf/contrib/tikz-cd/tikz-cd-doc.pdf
\usepackage{tikz-cd}

% The pdfrender package is used to output fonts with a larger stroke width.
\usepackage{pdfrender}

% The mathrsfs package provides the script font.
\usepackage{mathrsfs}

\usepackage[margin=0.6in]{geometry}

% The chngcntr ("change counter") package is used here so that subsection
% numbers are written without the leading section number. This takes place in
% the subsection headings as well as the theorem environment numbering.
%
% Before:
% 1. Section
% 1.1. Subsection
% Problem 1.1.1. What is 1 + 1?
% Problem 1.1.2. What is 1 + 2?
%
% After:
% 1. Section
% 1. Subsection
% Problem 1.1. What is 1 + 1?
% Problem 1.2. What is 1 + 2?
\usepackage{chngcntr}
\counterwithout{subsection}{section}

% The problem environment is a regular ams theorem environment with "Problem"
% text and some leading space to give some separation between the problems.
\theoremstyle{definition}
\newtheorem{problem-internal}{Problem}[subsection]
\newenvironment{problem}{
  \medskip
  \begin{problem-internal}
}{
  \end{problem-internal}
}

% The solution environment is a proof environment with the "solution" text as
% well as the following adjustments:
% - No indent on paragraphs;
% - A small amount of space between paragraphs.
%
% Note: The negative space at the beginning is to remove the space before the
% first paragraph in the solution.
\newenvironment{solution}{
  \begin{proof}[Solution]
  \vspace{-8px}
  \setlength{\parskip}{4px}
  \setlength{\parindent}{0px}
}{
  \end{proof}
}

% Renewing the \thesection command changes the section numbers to roman
% numerals. This matches the style of the Aluffi textbook.
%
% Before:
% 1. Section
% 1.1. Subsection
%
% After:
% I. Section
% I.1. Subsection
\renewcommand{\thesection}{\Roman{section}} 

% Renewing the \qedsymbol command changes the QED symbol. This QED symbol is
% automatically used by the proof environment.

% Set with spacing padding for the curly braces.
\newcommand{\set}[1]{\left\{\,#1\,\right\}}
\newcommand{\id}{\mathrm{id}}
\newcommand{\im}{\mathrm{im}}
\newcommand{\Obj}{\mathrm{Obj}}
\newcommand{\Hom}{\mathrm{Hom}}
\newcommand{\abs}[1]{\left|#1\right|}
% Inuitive "from" command draws an arrow pointing left, A <- B reads "A from p"
\newcommand{\from}{\leftarrow}
\newcommand{\quotuniv}[1]{\overline{#1}}

\begin{document}



\section*{Chapter 1}
\subsection*{Preliminaries: Set theory and categories}
\setcounter{subsection}{1}


% Problem 1.1
\begin{problem}
Locate a discussion of Russel's paradox, and understand it.
\end{problem}
\begin{solution}
Recall that, in naive set theory, any collection of objects
that satisfy some property can be called a set. Russel's paradox can be
illustrated as follows.  Let $R$ be the set of all sets that do not contain
themselves. Then, if $R\notin R$, then by definition it must be the case that
$R\in R$; similarly, if $R\in R$ then it must be the case that $R\notin R$.
\end{solution}


% Problem 1.2
\begin{problem}
$\rhd$ Prove that if $\sim$ is an equivalence relation on a set $S$, then
the corresponding family $\mathscr{P}_{\sim}$ defined in \S1.5 is indeed a
partition of $S$; that is, its elements are nonempty, disjoint, and their union
is $S$. [\S1.5]
\end{problem}

\begin{solution}
Let $S$ be a set with an equivalence relation $\sim$.
Consider the family of equivalence classes w.r.t. $\sim$ over $S$:
%
\[ \mathscr{P}_{\sim} = \set{[a]_{\sim} \mid a\in S} \]
%
Let $[a]_{\sim}\in\mathscr{P}_{\sim}$. Since $\sim$ is an equivalence relation,
by reflexivity we have $a\sim a$, so $[a]_{\sim}$ is nonempty. Now, suppose
$a$ and $b$ are arbitrary elements in $S$ such that $a\not\sim b$. For
contradiction, suppose that there is an $x\in [a]_{\sim}\cap[b]_{\sim}$. This
means that $x\sim a$ and $x\sim b$. By transitivity, we get that $a\sim b$; this
is a contradiction. Hence the $[a]_{\sim}$ are disjoint. Finally, let $x\in S$.
Then $x\in[x]_{\sim}\in \mathscr{P}_{\sim}$. This means that
%
\[ \bigcup_{[a]_{\sim} \in \mathscr{P}_{\sim}} [a]_{\sim} = S, \]
%
that is, the union of the elements of $\mathscr{P}_{\sim}$ is $S$.
\end{solution}


% Problem 1.3
\begin{problem}
$\rhd$ Given a partition $\mathscr{P}$ on a set $S$, show how to define a
relation $\sim$ such that $\mathscr{P} = \mathscr{P}_{\sim}$. [\S1.5]
\end{problem}

\begin{solution}
Define, for $a,b\in S$, $a\sim b$ if and only if there exists
an $X\in\mathscr{P}$ such that $a\in X$ and $b\in X$. We will show that
$\mathscr{P} = \mathscr{P}_{\sim}$. 

\begin{enumerate}
\item ($\mathscr{P}\subseteq\mathscr{P}_{\sim}$). Let $X\in \mathscr{P}$; we want to
show that $X\in\mathscr{P}_{\sim}$. We know that $X$ is
nonempty, so choose $a\in X$ and consider $[a]_{\sim}\in\mathscr{P}_{\sim}$. We
need to show that $X=[a]_{\sim}$. Suppose $a'\in X$ (it may be that $a'=a$.)
Since $a,a'\in X$, $a\sim a'$, so $a'\in[a]_{\sim}$. Now, suppose $a'\in
[a]_{\sim}$. We have $a'\sim a$, so $a'\in X$. Hence $X=[a]_{\sim}\in
\mathscr{P}_{\sim}$, so $\mathscr{P}\subseteq\mathscr{P}_{\sim}$.

\item ($\mathscr{P}_{\sim}\subseteq\mathscr{P}$). Let $[a]_{\sim}\in\mathscr{P}_{\sim}$.
From exercise I.1.1 we know that $[a]_{\sim}$ is non-empty. Suppose
$a'\in[a]_{\sim}$. By definition, since $a'\sim a$, there exists a set $X$ such
that $a,a'\in X$. Hence $[a]_{\sim}\subseteq X$. Also, if $a,a'\in X$ (not
necessarily distinct) then $a\sim a'$. Therefore,
$\mathscr{P}_{\sim}\subseteq\mathscr{P}$, and with 1. we get that the sets
$\mathscr{P}$ and $\mathscr{P}_{\sim}$ are equal.
\end{enumerate}
\end{solution}


% Problem 1.4
\begin{problem}
How many different equivalence relations can be defined on the set $\{1,2,3\}?$
\end{problem}

\begin{solution}
From the definition of an equivalence relation and the solution to problem
\textbf{I.1.3}, we can see that an equivalence relation on $S$ is equivalent to
a partition of $S$. Thus the number of equivalence relations on $S$ is equal to
the number of partitions of $S$. Since $\{1,2,3\}$ is small we can determine
this by hand:
%
\[ \mathscr{P}_0 = \set{\{1,2,3\}} \]
\[ \mathscr{P}_1 = \set{\{1\},\{2\},\{3\}\}} \]
\[ \mathscr{P}_2 = \set{\{1,2\},\{3\}} \]
\[ \mathscr{P}_3 = \set{\{1\},\{2,3\}} \]
\[ \mathscr{P}_4 = \set{\{1,3\},\{2\}} \]
%
Thus there can be only $5$ equivalence relations defined on $\{1,2,3\}$.
\end{solution}


% Problem 1.5
\begin{problem}
Give an example of a relation that is reflexive and symmetric but not
transitive. What happens if you attempt to use this relation to define a
partition on the set? (Hint: Thinking about the second question will help you
answer the first one.)
\end{problem}

\begin{solution}
For $a,b\in \mathbf{Z}$, define $a\diamond b$ to be true if and only if
$\abs{a-b} \leq 1$. It is reflexive, since $a\diamond a = \abs{a-a} = 0 \leq 1$
for any $a\in \mathbf{Z}$, and it is symmetric since $a\diamond b = \abs{a-b} =
\abs{b-a} = b\diamond a$ for any $a,b\in \mathbf{Z}$. However, it is not
transitive. Take for example $a=0, b=1, c=2$.  Then we have $\abs{a-b} = 1\leq
1$, and $\abs{b-c} = 1\leq 1$, but $\abs{a-c} = 2 > 1$; so $a\diamond b$ and
$b\diamond c$, but not $a\diamond c$.

When we try to build a partition of $\mathbf{Z}$ using $\diamond$, we get
"equivalence classes" that are not disjoint. For example, $[2]_{\diamond} =
\{1,2,3\}$, but $[3]_{\diamond} = \{2,3,4\}$. Hence $\mathscr{P}_{\diamond}$ is
not a partition of $\mathbf{Z}$.
\end{solution}


% Problem 1.6
\begin{problem}
Define a relation $\sim$ on the set $\mathbf{R}$ of real numbers by
setting $a\sim b\iff b-a\in\mathbf{Z}$. Prove that this is an equivalence
relation, and find a `compelling' description for $\mathbf{R}\,/\!\sim$. Do the
same for the relation $\approx$ on the plane $\mathbf{R}\times\mathbf{R}$ by
declaring $(a_1,b_1)\approx(a_2,b_2)\iff b_1-a_1\in\mathbf{Z}$ and
$b_2-a_2\in\mathbf{Z}$. [\S II.8.1, II.8.10]
\end{problem}

\begin{solution}
Suppose $a,b,c\in\mathbf{R}$. We have that $a-a=0\in\mathbf{Z}$, so $\sim$ is
reflexive. If $a\sim b$, then $b-a=k$ for some $k\in\mathbf{Z}$, so
$a-b=-k\in\mathbf{Z}$, hence $b\sim a$. So $\sim$ is symmetric. Now, suppose that
$a\sim b$ and $b\sim c$, in particular that $b-a=k\in\mathbf{Z}$ and
$c-b=l\in\mathbf{Z}$. Then $c-a=(c-b) + (b-a) = l+k\in\mathbf{Z}$, so $a\sim c$.
So $\sim$ is transitive.

An equivalence class $[a]_{\sim}\in\mathbf{R}\,/\!\sim$ is the set of integers
$\mathbf{Z}$ transposed by some real number $\epsilon\in[0,1)$. That is, for
every set $X\in\mathbf{R}\,/\!\sim$, there is a real number $\epsilon\in[0,1)$
such that every $x\in X$ is of the form $k+\epsilon$ for some integer $k$.

Now we will show that $\approx$ is an equivalence relation over
$\mathbf{R}\times\mathbf{R}$. Supposing $a_1,a_2\in\mathbf{R}\times\mathbf{R}$,
we have $a_1-a_1=a_2-a_2=0\in\mathbf{Z}$, so $(a_1,a_2)\approx(a_1,a_2)$. If we
also suppose that $b_1,b_2,c_1,c_2\in\mathbf{R}\times\mathbf{R}$, then symmetry
and transitivity can be shown as well: $(a_1,a_2)\approx(b_1,b_2)\implies
b_1-a_1=k$ for some integer $k$ and $b_2-a_2=l$ for some integer $l$, hence
$a_1-b_1=-k\in\mathbf{Z}$ and $a_2-b_2=-l\in\mathbf{Z}$, so
$(b_1,b_2)\approx(a_1,a_2)$; also if $(a_1,a_2)\approx(b_1,b_2)$ and
$(b_1,b_2)\approx(c_1,c_2)$, then
$(b_1,b_2)-(a_1,a_2)=(k_1,k_2)\in\mathbf{Z}\times\mathbf{Z}$ as well as
$(c_1,c_2)-(b_1,b_2)=(l_1,l_2)\in\mathbf{Z}\times\mathbf{Z}$, so $(c_1,c_2) -
(a_1,a_2) = (c_1,c_2) - (b_1,b_2) + (b_1,b_2) - (a_1,a_2) = (k_1+l_1,
k_2+l_2\in\mathbf{Z}\times\mathbf{Z}$. Thus $\approx$ is an equivalence
relation.

The interpretation of $\approx$ is similar to $\sim$. An equivalence class
$X\in\mathbf{R}\times\mathbf{R}\,/\approx$ is just the 2-dimensional integer
lattice $\mathbf{Z}\times\mathbf{Z}$ transposed by some pair of values
$(\epsilon_1,\epsilon_2)\in[0,1)\times[0,1)$.
\end{solution}



\subsection{Functions between sets}


% Problem 2.1
\begin{problem}
How many different bijections are there between a set with $n$ elements
and itself?
\end{problem}

\begin{solution}
A function $f:S\to S$ is a graph $\Gamma_f\subseteq S\times S$. Since $f$ is
bijective, then for all $y\in S$ there exists a unique $x\in S$ such that
$(x,y)\in\Gamma_f$. We can see that $\abs{\Gamma_f} = n$. Since each $x$ must be
unique, all the elements $x\in S$ must be present in the first component of
exactly one pair in $\Gamma_f$. Furthermore, if we order the elements $(x,y)$ in
$\Gamma_f$ by the first component, we can see that $\Gamma_f$ is just a
permutation on the $n$ elements in $S$. For example, for $S=\{1,2,3\}$ one such
$\Gamma_f$ is:
%
\[ \set{ (1,3), (2,2), (3,1) } \]
%
Since $\abs{S} = n$, the number of permutations of $S$ is $n!$. Hence there can be
$n!$ different bijections between $S$ and itself.
\end{solution}


% Problem 2.2
\begin{problem}
$\rhd$ Prove statement (2) in Proposition 2.1. You may assume that given a
family of disjoint subsets of a set, there is a way to choose one element in
each member of the family. [\S2.5, V3.3]

\begin{quote} \textbf{Proposition 2.1.} Assume $A\neq\emptyset$, and let $f:A\to
B$ be a function. Then
(1) $f$ has a left-inverse if and only if $f$ is injective; and \\
(2) $f$ has a right-inverse if and only if $f$ is surjective.
\end{quote}
\end{problem}

\begin{solution}
Let $A\neq\emptyset$ and suppose $f:A\to B$ is a function.

($\implies$) Suppose there exists a function $g$ that is a right-inverse of $f$.
Then $f\circ g = \id_B$. Let $b\in B$. We have that $f(g(b)) = b$, so there
exists an $a = g(b)$ such that $f(a) = b$. Hence $f$ is surjective.

($\impliedby$) Suppose that $f$ is surjective. We want to construct a function
$g:B\to A$ such that $f(g(a)) = a$ for all $a\in A$. Since $f$ is surjective,
for all $b\in B$ there is an $a\in A$ such that $f(a) = b$. For each $b\in B$
construct a set $\Lambda_b$ of such pairs:
%
\[ \Lambda_b = \set{ (a,b) \mid a \in A, f(a) = b } \]
%
Note that $\Lambda_b$ is non-empty for all $b\in B$. So that we can choose one
pair $(a,b)$ ($a$ not necessarily unique) from each set in $\Lambda =
\set{\Lambda_b\mid b\in B}$ to define $g:B\to A$:
%
\[ g(b) = a, \text{ where $a$ is in some $(a,b)\in\Lambda_b$} \]
%
Now, $g$ is a right-inverse of $f$. To show this, let $b\in B$. Since $f$ in
surjective, $g$ has been defined such that when $a=g(b)$, $f(a)=b$, so we get
that $f(g(b)) = (f\circ g)(b) = b$, thus $g$ is a right-inverse of $f$.
\end{solution}


% Problem 2.3
\begin{problem}
Prove that the inverse of a bijection is a bijection and that the
composition of two bijections is a bijection.
\end{problem}

\begin{enumerate}
\item Suppose $f:A\to B$ is a bijection, and that $f^{-1}:B\to A$ is its inverse.
We have that $f\circ f^{-1} = \id_B$ and $f^{-1}\circ f = \id_A$. Hence $f$ is
the left- and right-inverse of $f^{-1}$, so $f^{-1}$ must be a bijection.

\item Let $f:B\to C$ and $g:A\to B$ be bijections, and consider $f\circ g$. To
show that $f$ is injective, let $a, a'\in A$ such that $(f\circ g)(a) = (f\circ
g)(a')$. Since $f$ is a bijection, $f(g(a)) = f(g(a')) \implies g(a) = g(a')$.
Also, since $g$ is a bijection, $g(a) = g(a') \implies a=a'$. Hence $f\circ g$
is injective. Now, let $c\in C$. Since $f$ is surjective, there is a $b\in B$
such that $f(b) = c$. Also, since $g$ is surjective, there is an $a\in A$ such
that $g(a) = b$; this means that there is an $a\in A$ such that $(f\circ g)(a) =
c$. So $f\circ g$ is bijective.
\end{enumerate}


% Problem 2.4
\begin{problem}
$\rhd$ Prove that `isomorphism' is an equivalence relation (on any set
of sets.) [\S4.1]
\end{problem}

\begin{solution}
Let $S$ be a set. Then $\id_S$ is a bijection from $S$ to itself, so $S\cong S$.
Let $T$ be another set with $S\cong T$, i.e. that there exists a bijection
$f:S\to T$. Since $f$ is a bijection, it has an inverse $f^{-1}:T\to S$, so
$T\cong S$. Finally, let $U$ also be a set, and assume that there exists
bijections $f:S\to T$ and $g:T\to U$, i.e. that $S\cong T$ and $T\cong U$. From
exercise \textbf{I.2.3} we know that the composition of bijections is itself a
bijection. This means that $g\circ f: S\to U$ is a bijection, so $S\cong U$.
Hence $\cong$ is an equivalence relation.
\end{solution}


% Problem 2.5
\begin{problem}
$\rhd$ Formulate a notion of \textit{epimorphism}, in the style
of the notion of \textit{monomorphism} seen in \S 2.6, and prove a result
analogous to Proposition 2.3, for epimorphisms and surjections.
\end{problem}

\begin{solution}
A function $f:A\to B$ is an \textit{epimorphism} if and only if for all sets $Z$
and all functions $b':Z\to B$, there is a function $a':Z\to A$ such that $f\circ
a' = b'$. Now we will show that $f$ is a surjection if and only if it is an
epimorphism.

($\implies$) Suppose that $f:A\to B$ is surjective. Let $Z$ be a set and
$b':Z\to B$ a function. We need to construct a function $a':Z\to A$ such that
$f\circ a' = b'$. Fix $z\in Z$. Suppose $b=b'(z)\in B$. Since $b\in B$ and $f$
is surjective, there exists an $a\in A$ such $f(a) = b$. Now, define $a'(z) =
a$ this way for each $z\in Z$. Then $f\circ a'(z) = b'(z)$ for all $z\in Z$, so
$f\circ a' = b'$. Hence $f$ is an epimorphism.

($\impliedby$) Suppose that $f$ is an epimorphism. Let $b':B\to B$ be a
bijection. Since $f$ is an epimophism, there is a function $a':B\to A$ such that
$f\circ a' = b'$. Let $b\in B$. Since $b'$ is a bijection, there is a unique
element $y\in B$ such that $b'(y) = b$. Furthermore, we have that $(f\circ
a')(y) = b$. In other words, $a = a'(y)$ is an element in $a$ such that $f(a) =
b$. Hence $f$ is surjective, as required.
\end{solution}


% Problem 2.6
\begin{problem}
With notation as in Example 2.4, explain how any function $f:A\to B$ determines
a section of $\pi_A$.
\end{problem}

\begin{solution}
Let $f:A\to B$ and let $\pi_A:A\times B\to A$ be such that $\pi_A(a,b) = a$ for
all $(a,b)\in A\times B$. Construct $g:A\to A\times B$ defined as $g(a) = (a,
f(a))$ for all $a\in A$. The function $g$ can be thought of as `determined by'
$f$. Now, since $(\pi_A\circ g)(a) = \pi_A(g(a)) = \pi_A(a, f(a)) = a$ for all
$a\in A$, $g$ is a right inverse of $\pi_A$, i.e. $g$ is a section of $\pi_A$ as
required.
\end{solution}


% Problem 2.7
\begin{problem}
Let $f:A\to B$ be any function. Prove that the graph $\Gamma_f$ of $f$ is
isomorphic to $A$.
\end{problem}

\begin{solution}
Recall that sets $\Gamma_A$ and $A$ are \textit{isomorphic}, written
$\Gamma_A\cong A$, if and only if there exists a bijection $g:\Gamma_A\to A$.
Let's construct such a function $g$, defined to be $g(a,b) = a$. Keep in mind
that here $(a,b)\in\Gamma_f\subseteq A\times B$.

Let $(a',b'),(a'',b'')\in\Gamma_f$ such that $f(a',b') = f(a'',b'')$. For
contradiction, suppose that $(a',b')\neq (a'',b'')$. Since $f(a',b') = a' = a''
= f(a'',b'')$, it must be that $b'\neq b''$. However, this would mean that both
$(a',b')$ and $(a',b'')$ are in $\Gamma_f$; this would mean that $f(a') = b'
\neq b'' = f(a')$, which is impossible since $f$ is a function. Hence $g$ is
injective.

Let $a'\in A$. Since $f$ is a well-defined function with $A$ as its domain,
there must exists a pair $(a',b')\in\Gamma_f$ for some $b'\in B$, in particular
that $g(a',b') = a'$; thus $g$ is surjective, so it is a bijection.
\end{solution}


% Problem 2.8.
\begin{problem}
Describe as explicitly as you can all terms in the canonical decomposition (cf.
\S2.8) of the function $\mathbf{R}\to\mathbf{C}$ defined by $r\mapsto e^{2\pi
ir}$. (This exercise matches one previously. Which one?)
\end{problem}

\begin{solution}
Let $f:\mathbf{R}\to\mathbf{C}$ be as above. The first piece in the canonical
decomposition is the equivalence relation $\sim$ defined as $x \sim x' \iff f(x) =
f(x')$, i.e. $[x]_{\sim}$ is the set of all elements in $\mathbf{R}$ that get
mapped to the same element in $\mathbf{C}$ by $f$ as $x$.

The second piece is the set $\mathscr{P}_{\sim}$. This set is the set of all
equivalence classes of $\mathbf{R}$ over equality up to $f$. Note that, since
$f(x) = e^{2\pi i x} = \cos(2\pi x) + i\sin(2\pi x)$, $f$ is periodic with
period $1$. That is, $f(x) = e^{2\pi i x} = e^{2\pi i x + 2\pi} = e^{2\pi i (x +
1)} = f(x+1)$. In other words, we can write $\mathscr{P}_{\sim}$ as,
%
\[ \mathscr{P}_{\sim} = \set{\set{r + k\mid k\in\mathbf{Z}}\mid
r\in[0,1)\subseteq\mathbf{R}}, \]
%
and it is here when we notice uncanny similarities to exercise \textbf{1.6}
where $x\sim y$, for $x,y\in\mathbf{R}$, if and only if $x-y\in\mathbf{Z}$, in
which we could have written $\mathscr{P}_{\sim}$ in the same way. 

Now we will explain the mysterious $\tilde{f}:\mathscr{P}_{\sim}\to\im f$. This
function is taking each \textit{equivalence class} $[x]_{\sim}$ over the reals
w.r.t. $\sim$ and mapping it to the element in $\mathbf{C}$ that $f$ maps each
element $x'\in[x]_{\sim}$ to; indeed, since $x\sim x'$ is true for
$x,x'\in\mathbf{R}$ if and only if $f(x)=f(x')$, we can see that for any
$x\in\mathbf{R}$, for all $x'\in[x]_{\sim}$, there exists a $c\in\mathbf{C}$
such that $f(x') = c$. To illustrate with the equivalence class over
$\mathbf{R}$ w.r.t. $\sim$ corresponding to the element $0\in\mathbf{R}$, we
have $[0]_{\sim} = \set{\dots, -2, -1, 0, 1, 2, \dots}$.  We can see that
$e^{-4\pi i} = e^{-2\pi i} = e^{0\pi i} = 1 = e^{2\pi i} = e^{4\pi i}$, etc; so
the function would map $[0]_{\sim}\mapsto1\in\mathbf{C}$, and so on.
Furthermore, we can see that $\tilde{f}$ is surjective, since for $y$ to be in
$\im f$ is to say that there is an $x\in\mathbf{R}$ such that $f(x) = y$; so
there must be an equivalence class $[x]_{\sim}$ which is mapped to $y$ by
$\tilde{f}$.

Finally, the simple map from $\im f\to\mathbf{C}$ that simply takes $c\mapsto
c$. This can be thought of as a potential ``expansion'' of the domain of
$\tilde{f}$. It is obviously injective, since (trivially) $c\neq c'\implies
c\neq c'$. However, it may not be surjective: for example, $2\in\mathbf{C}$ is
not in $\im f$ as it is defined above.
\end{solution}


% Problem 2.9
\begin{problem}
$\rhd$ Show that if $A'\cong A''$ and $B'\cong B''$, and further
$A'\cap B'=\emptyset$ and $A''\cap B''=\emptyset$, then $A'\cup B'\cong A''\cup
B''$. Conclude that the operation $A\amalg B$ is well-defined up to
\textit{isomorphism} (cf. \S2.9) [\S2.9, 5.7]
\end{problem}

\begin{solution}
Let $A',A'',B',B''$ be sets as described above. Since $A'\cong A''$ and $B'\cong
B''$, we know there exists respective bijections $f:A'\to A''$ and $g:B'\to
B''$. Now, we wish to show that $A'\cup B'\cong A''\cup B''$. Define a function
$h:A'\cup B'\to A''\cup B''$ such that $h(x) = f(x)$ if $x\in A'$ and $g(x)$ if
$x\in B'$.

We will now show that $h$ is a bijection. Let $y\in A''\cup B''$. Then, since
$A''\cap B''=\emptyset$, either $y\in A''$ or $y\in B''$. Without loss of
generality suppose that $y\in A''$. Then, since $f:A'\to A''$ is a bijection, it
is \textit{surjective}, so there exists an $x\in A'\subseteq A'\cup B'$ such
that $h(x) = f(x) = y$. So $h$ is surjective. Now, suppose that $x\neq x'$, for
$x,x'\in A'\cup B'$. If $x,x'\in A'$, then since $f$ is injective and $h(x) =
f(x)$ for all $x\in A'$, then $h(x)\neq h(x')$. Similarly for if $x,x'\in B'$.
Now, without loss of generality if $x\in A'$ and $x'\in B'$, then $h(x) = f(x)
\neq g(x') = h(x')$ since $A''\cap B''=\emptyset$. Hence $h$ is a bijection, so
$A'\cup B'\cong A''\cup B''$.

Since these constructions of $A',A'',B',B''$ correspond to creating ``copies''
of sets $A$ and $B$ for use in the disjoint union operation, we have that
disjoint union is a well-defined function \textit{up to isomorphism}. In
particular, since $\cong$ is an equivalence relation, we can consider $\amalg$
to be well-defined from $\mathscr{P}_{\cong}$ to $A'\cup B'$.
\end{solution}


% Problem 2.10
\begin{problem}
$\rhd$ Show that if $A$ and $B$ are finite sets, then $\abs{B^A} =
\abs{B}^{\abs{A}}$. [\S2.1, 2.11, I.4.1]
\end{problem}

\begin{solution}
Let $A$ and $B$ be sets with $\abs{A}=n$ and $\abs{B}=m$, with $n,m$ being
non-negative integers. Recall that $B^A$ denotes the set of functions $f:A\to
B$. Now, if $A=B=\emptyset$ or $A=\emptyset$ and $\abs{B}=1$, we get one
function, the empty function $\Gamma_f = \emptyset$, and $0^0 = 1^0 = 1$. If
$\abs{A} = \abs{B} = 1$, then we get the singleton function
$\Gamma_f=\{(a,b)\}$, and $1^1 = 1$. If $A\neq\emptyset$ and $B=\emptyset$, then
no well-defined function can exist from $A$ to $B$ since there will be no value
for the elements in $A$ to take; this explains $\abs{B^A} = \abs{B}^{\abs{A}} =
0^{\abs{A}} = 0$.

Suppose that $B\neq\emptyset$ and $B$ is finite. We will show inductively that
$\abs{B^A} = \abs{B}^{\abs{A}}$. First, suppose that $\abs{A} = 1$.  Then there
are exactly $\abs{B}$ functions from $A$ to $B$: if $B=\set{b_1,b_2,\dots,b_m}$,
then the functions are $\{(a,b_1)\}, \{(a,b_2)\}$, etc. Hence $\abs{B^A} =
\abs{B}^{\abs{A}} = \abs{B}$.  Now, fix $k\geq 2$, and assume that $\abs{B^A} =
\abs{B}^{\abs{A}}$ for all sets $A$ such that $\abs{A}=k-1$. Suppose that
$\abs{A}=k$. Let $a\in A$. (We can do this since $\abs{A}=k\geq 2$.) Then, by
the inductive hypothesis, since $\abs{A\backslash\{a\}}=k-1$,
$\abs{B^{(A\backslash\{a\})}} = \abs{B}^{\abs{A}-1}$. Let $F$ be the set of
functions from $A\backslash\{a\}$ to $B$.  Then, for each of those functions
$f\in F$, there is $\abs{B}$ ``choices'' of where to assign $a$: one choice for
each element in $B$. Hence, $\abs{B^A} = \abs{B}\abs{B}^{\abs{A}-1} =
\abs{B}^{\abs{A}}$ as required.
\end{solution}


% Problem 2.11
\begin{problem}
$\rhd$ In view of Exercise 2.10, it is not unreasonable to use $2^A$ to denote
the set of functions from an arbitrary set $A$ to a set with $2$ elements (say
$\{0,1\}$). Prove that there is a bijection between $2^A$ and the \textit{power
set} of $A$ (cf. \S1.2). [\S1.2, III.2.3]
\end{problem}

\begin{solution}
Let $S = \{0,1\}$, and consider $f:\mathcal{P}(A)\to 2^A$, defined as
%
\[ f(X) = \set{(a,1) \text{ if $a\in X$, and }(a,0) \text{ otherwise}} \]
%
We will show that $f$ is bijective. Let $g\in 2^A$. Then $f$ is a
function from $A$ to $S$. Let $A_1 = \set{a\in A\mid g(a) = 1}$. Then $A_1$ is a
set such that $A_1\in\mathcal{P}(A)$, and $f(A_1)=g$. Hence $f$ is surjective.

Now, suppose that $X,Y\subseteq A$ and $f(X) = f(Y)$. Then, for all $a\in A$,
$a\in X \iff f(X)(a) = 1 \iff f(Y)(a) = 1 \iff a\in Y$. Hence $f$ is injective,
so $2^A\cong\mathcal{P}(A)$.
\end{solution}



\subsection{Category theory}


% Problem 3.1
\begin{problem}
$\rhd$ Let $\mathsf{C}$ be a category. Consider a structure
$\mathsf{C}^{op}$ with
\begin{enumerate}
\item $\Obj(\mathsf{C}^{op}) = \Obj(\mathsf{C})$
\item For $A, B$ objects of $\mathsf{C}^{op}$ (hence objects of $\mathsf{C}$),
$\Hom_{\mathsf{C}^{op}}(A, B) := \Hom_{\mathsf{C}}(B, A)$.
\end{enumerate}
Show how to make this into a category (that is, define composition of morphisms
in $\mathsf{C}^{op}$ and verify the properties listed in \S3.1).

Intuitively, the `opposite' category $\mathsf{C}^{op}$ is simply obtained by
`reversing all the arrows' in $\mathsf{C}$. [5.1, \S III.1.1, \S IX.1.2,
IX.1.10]
\end{problem}

\begin{solution}
For objects $A,B,C\in\Obj(\mathsf{C}^{op})$, the set of morphisms between $A$
and $B$ in $\mathsf{C}^{op}$, $\Hom_{\mathsf{C}^{op}}(A,B)$, is defined as
$\Hom_{\mathsf{C}}(B, A)$. Similarly for the morphisms between $B$ and $C$. So
for morphisms $f\in\Hom_{\mathsf{C}^{op}}(A,B)$ and
$g\in\Hom_{\mathsf{C}^{op}}(B,C)$, to define composition we recall the
set-function $\circ_{\mathsf{C}} : \Hom_{\mathsf{C}}(C,B) \times
\Hom_{\mathsf{C}}(B,A) \to\Hom_{\mathsf{C}}(C,A)$ that is defined for the
objects $A,B,C\in\Obj(\mathsf{C})=\Obj(\mathsf{C}^{op})$; we shall define the
composition of morphisms $f:A\to B$ and $g:B\to C$ in $\mathsf{C}^{op}$ with
this function. Precisely, we define
%
\[ \circ_{\mathsf{C}^{op}}: \Hom_{\mathsf{C}^{op}}(A,B) \times
\Hom_{\mathsf{C}^{op}}(B,C) \to \Hom_{\mathsf{C}^{op}}(A,C) \]
%
to be
%
\[ \circ_{\mathsf{C}^{op}}(f,g) = \circ_{\mathsf{C}}(g,f) \]
%
for all $f\in\Hom_{\mathsf{C}^{op}}(A,B)$ and $f\in\Hom_{\mathsf{C}^{op}}(B,C)$.
The domain and codomain of $\circ_{\mathsf{C}}$ and $\circ_{\mathsf{C}^{op}}$
match (up to transposing the coordinates in the domain) due to the equality of
$\Hom_{\mathsf{C}}(A,B)$ with $\Hom_{\mathsf{C}^{op}}(B,A)$.

To show that this composition makes $\mathsf{C}^{op}$ a category, first we note
that the fact that $\mathsf{C}$ is a category implies the existence of a
morphism $1_{A}$ taking $A$ to itself where $A\in\Obj(\mathsf{C})$; this
morphism is thus also present in $\Hom_{\mathsf{C}^{op}}(A,A) =
\Hom_{\mathsf{C}}(A,A)$. Secondly, for objects $A,B,C,D\in\Obj(\mathsf{C})$, any
morphisms $f\in\Hom_{\mathsf{C}^{op}}(A,B)$, $g\in\Hom_{\mathsf{C}^{op}}(B,C)$,
and $h\in\Hom_{\mathsf{C}^{op}}(C,D)$ are associative, since
%
\[ (h \circ_{\mathsf{C}^{op}} g) \circ_{\mathsf{C}^{op}} f = f
\circ_{\mathsf{C}} (g \circ_{\mathsf{C}} h) = (f \circ_{\mathsf{C}} g)
\circ_{\mathsf{C}} h = h \circ_{\mathsf{C}^{op}} (g \circ_{\mathsf{C}^{op}} f).
\]
%
Finally, for any morphism $f\in\Hom_{\mathsf{C}^{op}}(A,B)$ we have,
%
\[ f \circ_{\mathsf{C}^{op}} 1_A = 1_A \circ_{\mathsf{C}} f = f \text{ and }
1_B \circ_{\mathsf{C}^{op}} f = f \circ_{\mathsf{C}} 1_B = f; \]
%
hence the identities are ``identities with respect to composition''. Last, for
objects $A,B,C,D\in\Obj(\mathsf{C})$ where $A\neq C$ and $B\neq D$,
clearly $\Hom_{\mathsf{C}}(B,A)\cap\Hom_{\mathsf{C}}(D,C)=\emptyset$ is true iff 
$\Hom_{\mathsf{C^{op}}}(A,B)\cap\Hom_{\mathsf{C^{op}}}(C,D)=\emptyset$. Hence
$\mathsf{C}^{op}$ is a category.
\end{solution}


% Problem 3.2
\begin{problem}
If $A$ is a finite set, how large is $\mathrm{End}_{\mathsf{Set}}(A)$?
\end{problem}

\begin{solution}
The set $\mathrm{End}_{\mathsf{Set}}(A)$ is the set of functions $f:A\to A$.
Since $A$ is finite, write $\abs{A} = n$ for some $n\in\mathbf{Z}$. By exercise
2.10, we know that $\abs{A^A} = \abs{A}^{\abs{A}} = n^n$. So the the set
$\mathrm{End}_{\mathsf{Set}}(A)$ has size $n^n$.
\end{solution}


% Problem 3.3
\begin{problem}
$\rhd$ Formulate precisely what it means to say that $1_a$ is an identity with
respect to composition in Example 3.3, and prove this assertion. [\S3.2]
\end{problem}

\begin{solution}
Let $S$ be a set and $\sim$ be a binary relation on the set $S$. Then, for
elements $a,b\in S$, $\Hom(a,b)$ is the pair $(a,b)\in S\times S$ if $a\sim b$,
or $\emptyset$ otherwise. Composition of morphisms $(a,b)$ and $(b,c)$ is simply
the pair $(a,c)$, which captures the transitivity of $\sim$. We will say that
$1_a = (a,a)$, for $a\in S$, is \textit{an identity with respect to composition}
if, for any $b\in S$, $(a,b)(a,a) = (a,b)$. Now, if $a\sim a$ and $a\sim b$,
then trivially it is the case that $a\sim b$; hence $(a,b)(a,a) = (a,b)$, and
$1_a$ is an identity w.r.t. composition as required.
\end{solution}


% Problem 3.4
\begin{problem}
Can we define a category in the style of Example 3.3 using the relation $<$ on
the set $\mathbf{Z}$?
\end{problem}

\begin{solution}
No, we can't. This is because $<$ isn't reflexive: $x\not<x$ for any
$x\in\mathbf{Z}$.
\end{solution}


% Problem 3.5
\begin{problem}
$\rhd$ Explain in what sense Example 3.4 is an instance of the categories
considered in Example 3.3. [\S 3.2]
\end{problem}

\begin{solution}
Let $S$ be a set. Example 3.4 considers the category $\hat{S}$ with objects
$\Obj(\hat{S}) = \mathscr{P}(S)$ and morphisms $\Hom_{\hat{S}}(A,B) =
\set{(A,B)}$ if $A\subseteq B$ and $\emptyset$ otherwise, for all sets
$A,B\in\mathscr{P}$. The category $\hat{S}$ is an instance of the categories
explained in Example 3.3 because $\subseteq$ is a reflexive and transitive
operation on the power set of any set $S$. Indeed, for $X,Y,Z\subseteq S$, we have
that $X\subseteq X$ and, if $X\subseteq Y$ and $Y\subseteq Z$, then if $x\in X$,
then $x\in Y$ and $x\in Z$ so $X\subseteq Z$.
\end{solution}


% Problem 3.6
\begin{problem}
$\rhd$ (Assuming some familiarity with linear algebra.) Define a category
$\mathsf{V}$ by taking $\Obj(\mathsf{V}) = \mathbf{N}$ and letting
$\Hom_{\mathsf{V}}(m,n) = $ the set of $m\times n$ matrices with real
entries, for all $m,n\in\mathbf{N}$. (We will leave the reader the task of
making sense of a matrix with 0 rows or columns.) Use product of matrices to
define composition. Does this category `feel' familiar? [\S VI.2.1, \S VIII.1.3]
\end{problem}

\begin{solution}
Yes! It is yet another instance of Example 3.3. The binary relation $\sim$ on
$\mathbf{N} \times \mathbf{N}$ holds for all values $(n,m)\in\mathbf{N} \times
\mathbf{N}$, and means that a matrix of size $m\times n$ ``can be built''. It is
reflexive trivially. It is transitive trivially as well---a matrix of any size
can be built. However, it would also hold, for example, if we had to in some
sense ``deduce'' that a $3\times 3$ matrix could be built using the fact that
$3\times 1$ and $1\times 3$ matrices can be built.
\end{solution}


% Problem 3.7
\begin{problem}
$\rhd$ Define carefully the objects and morphisms in Example 3.7, and draw the
diagram corresponding to compositon. [\S 3.2]
\end{problem}

\begin{solution}
\def \C {\mathsf{C}}
\def \CA {\mathsf{C}^A}

Let $\C$ be a category, and $A\in\C$. We want to define $\CA$. Let $\Obj(\CA)$
include all morphisms $f\in\Hom_\C(A,Z)$ for all $Z\in\Obj(\C)$. For any two
objects $f,g\in\Obj(\CA)$, $f:A\to Z_1$ and $g:A\to Z_2$, we define the
morphisms $\Hom_\CA(f,g)$ to be the morphisms $\sigma\in\Hom_\C(Z_1, Z_2)$ such
that $g=\sigma f$. Now we must check that these morphisms satisfy the axioms.

\begin{enumerate}
\item Let $f\in\Obj(\CA)\in\Hom_\C(A,Z)$ for some object $Z\in\Obj(\C)$. Then
there exists an identity morphism $1_Z\in\Hom_\C(Z,Z)$ since $\C$ is a category.
This is a morphism such that $f=1_zf$, so $\Hom_\CA(f,f)$ is also nonempty.

\item Let $f,g,h\in\Obj(\CA)$ such that there are morphisms
$\sigma\in\Hom_\CA(f,g)$ and $\tau\in\Hom_\CA(g,h)$. Then there is a morphism
$\upsilon\in\Hom_\CA(f,h)$, namely $\tau\sigma$, which exists because of
morphism composition in $\C$. For clarity, we write that  $f:A\to Z_1$, $g:A\to
Z_2$, $h:A\to Z_3$, with $\sigma:Z_1\to Z_2$ and $\tau:Z_2\to Z_3$. We have
$g=\sigma f$ and $h=\tau g$. Hence, $\upsilon f = \tau\sigma f = \tau g = h$ as
required.

\item Lastly, let $f,g,h,i\in\Obj(\CA)$ with $Z_1, Z_2, Z_3, Z_4$ codomains
respectively, and with $\sigma\in\Hom_\CA(f,g)$, $\tau\in\Hom_\CA(g,h)$, and
$\upsilon\in\Hom_\CA(h,i)$. Since $\sigma$, $\tau$, and $\upsilon$ are morphisms
in $\C$ taking $Z_1\to Z_2$, etc., morphism composition is associative; hence
morphism composition is associative in $\CA$ as well.
\end{enumerate}
\end{solution}


% Problem 3.8
\begin{problem}
\def \C {\mathsf{C}}
\def \Cp {\C'}

$\rhd$ A \textit{subcategory} $\Cp$ of a category $\C$ consists of a
collection of objects of $\C$, with morphisms
$\Hom_\Cp(A,B) \subseteq \Hom_\C(A,B)$ for all objects $A,B\in\Obj(\Cp)$, such
that identities and compositions in $\C$ make $\Cp$ into a category. A
subcategory $\Cp$ is \textit{full} if $\Hom_\Cp(A,B) = \Hom_\C(A,B)$ for all
$A,B\in\Obj(\Cp)$. Construct a category of \textit{infinite sets} and explain
how it may be viewed as a full subcategory of $\mathsf{Set}$. [4.4,\S VI.1.1, \S
VIII.1.3]
\end{problem}

\begin{solution}
\def \C {\mathsf{C}}
\def \Cp {\C'}
\def \Set {\mathsf{Set}}
\def \ISet {\mathsf{Inf}\mathsf{Set}}

Let $\ISet$ be a subcategory of $\Set$ with $\Obj(\ISet)$ being all infinite
sets and $\Hom_\ISet(A,B)$ for infinite sets $A,B$ being the functions from $A$
to $B$. Since $\Hom_\Set(A,B)$ is just the set of all functions from $A$ to $B$
and not, say, the set of all functions from subsets of $A$ that are in
$\Obj(\Set)$ to $B$, $\ISet$ is full since $\Hom_\ISet(A,B)=\Hom_\Set(A,B)$ for
all infinite sets $A,B\in\Obj(\ISet)$.
\end{solution}


% Problem 3.9
\begin{problem}
\def \Set {\mathsf{Set}}
\def \MSet {\mathsf{MSet}}

$\rhd$ An alternative to the notion of \textit{multiset} introduced in
\S2.2 is obtained by considering sets endowed with equivalence relations;
equivalent elements are taken to be multiple instance of elements `of the same
kind'. Define a notion of morphism between such enhanced sets, obtaining a
category $\MSet$ containing (a `copy' of) $\Set$ as a full subcategory. (There
may be more than one reasonable way to do this! This is intentionally an
open-ended exercise.) Which objects in $\MSet$ determine ordinary multisets as
defined in \S2.2 and how? Spell out what a morphism of multisets would be from
this point of view. (There are several natural motions of morphisms of
multisets. Try to define morphisms in $\MSet$ so that the notion you obtain for
ordinary multisets captures your intuitive understanding of these objects.)
[\S2.2, \S3.2, 4.5]
\end{problem}

\begin{solution}
\def \Set {\mathsf{Set}}
\def \MSet {\mathsf{MSet}}

Define $\Obj(\MSet)$ as all tuples $(S, \sim)$ where $S$ is a set and $\sim$ is
an equivalence relation on $S$. For two multisets $\hat{S} = (S,\sim), \hat{T} =
(T,\approx) \in \Obj(\MSet)$, we define a morphism
$f\in\Hom_\MSet(\hat{S},\hat{T})$ to be a set-function $f:S\to T$ such that, for
$x,y\in S$, $x\sim y\implies f(x)\approx f(y)$, and morphism composition the
same way as set-functions. Now we verify the axioms:

\begin{enumerate}
\item For a multiset $(S,\sim)$, we borrow the set-function $1_S:S\to S$ and
note that it necessarily preserves equivalence, i.e. $x\sim y\implies 1_S(x)\sim
1_S(y)$.
\item Let there be objects $\hat{S}=(S,\sim), \hat{T}=(T,\approx),
\hat{U}=(U,\cong)$ with morphisms $f\in\Hom_\MSet(\hat{S},\hat{T})$ and
$g\in\Hom_\MSet(\hat{T},\hat{U})$. Note that $gf:S\to U$ is a set-function since
$\Set$ is a category. Now, since $f$ is a morphism in $\MSet$, for
$x,y\in S$, if $x\sim y$, then $f(x)\approx f(y)$, and since $f(x),f(y)\in T$
and $g$ is a morphism in $\MSet$, $g(f(x))\cong g(f(y))$.
\item Associativity can be proven similarly.
\end{enumerate}

Hence $\MSet$ as defined above is a category. Now, recall that multisets are
defined in \S2.2 as a set $S$ and a \textit{multiplicity function}
$m:S\to\mathbf{N}$. So, for any set $S$ and function $m:S\to\mathbf{N}$, if we
define the equivalence relation corresponding to $m$ as $\sim_m$ then the
tuple $(S,\sim_m)\in\Obj(\MSet)$. The objects in $\MSet$ which
\textit{don't} correspond to any multiset as defined in \S2.2 are sets $S$ with
equivalence relations $\sim$ such that both $S$ and $\mathscr{P}_\sim$ are
uncountable; this way, one cannot construct a function $m:S\to\mathbf{N}$
corresponding to each set in the partition $\mathscr{P}_\sim$, since
$\mathbf{N}$ is countable.
\end{solution}


% Problem 3.10
\begin{problem}
\def \C {\mathsf{C}}
\def \Set {\mathsf{Set}}

Since the objects of a category $\C$ are not (necessarily) sets, it is not clear
how to make sense of a notion of `subobject' in general. In some situations it
\textit{does} make sense to talk about subobjects, and the subobjects of any
given object $A$ in $\C$ are in one-to-one correspondence with the morphisms
$A\to\Omega$ for a fixed, special object $\Omega$ of $\C$, called a
\textit{subobject classifier}. Show that $\Set$ has a subobject classifier.
\end{problem}

\begin{solution}
\def \C {\mathsf{C}}
\def \Set {\mathsf{Set}}

Let $A\in\Obj(\Set)$. Any set $X\subseteq A$ corresponds to a mapping
$A\to\{0,1\}$; the elements $x\in A$ that are also in $X$ are mapped to $1$, and
the elements $x\in A$ that aren't in $X$ are mapped to $0$. Hence the
``subobject classifier'' for $\Set$ is $\Omega=\{0,1\}$.
\end{solution}


% Problem 3.11
\begin{problem}
\def \C {\mathsf{C}}

$\rhd$ Draw the relevant diagrams and define composition and identities for the
category $\C^{A,B}$ mentioned in Example 3.9. Do the same for the category
$\C^{\alpha,\beta}$ mentioned in Example 3.10. [\S5.5, 5.12]
\end{problem}

\begin{solution}
\def \C {\mathsf{C}}

Let $\C$ be a category, with $A,B\in\Obj(\C)$. The objects of $\C^{A,B}$ are
then diagrams:
%
\[\begin{tikzcd}[row sep=tiny]
A \arrow[dr, "f"]  &   \\
                   & Z \\
B \arrow[ur, "g"'] & 
\end{tikzcd}\]
%
Namely, tuples $(Z,f,g)$ where $Z\in\Obj(\C),g\in\Hom_\C(A,Z)$, and
$f\in\Hom_\C(B,Z)$. For objects $O_1=(Z_1,f_1,g_1)$ and $O_2=(Z_2,f_2,g_2)$ in
$\Obj(C^{A,B})$, the morphisms between them are morphisms
$\sigma\in\Hom_\C(Z_1,Z_2)$ such that $\sigma f_1=f_2$ and $\sigma g_1=g_2$.
This forms the following commutative diagram:
%
\[\begin{tikzcd}
A \arrow[dr, "f_1"] \arrow[drr, bend left, "f_2" near end]    &                               \\
                                                     & Z_1 \arrow[r, "\sigma"] & Z_2 \\
B \arrow[ur, "g_1"'] \arrow[urr, bend right, "g_2"' near end] &
\end{tikzcd}\]
%
Given a third object $O_3=(Z_3,f_3,g_3)$, with another morphism $\tau:O_2\to
O_3$ (which is a morphism from $Z_2\to Z_3$), composition in $\C^{A,B}$ is
defined the same way as composition in $\C$: $\tau\sigma:Z_1\to Z_3$. Since
$\sigma$ and $\tau$ both commute (i.e.  $\sigma f_1=f_2$, $\sigma g_1=g_2$,
$\tau f_2=f_3$, and $\tau g_2=g_3$), then $\tau\sigma$ also commutes: $\tau
\sigma f_1=\tau f_2=f_3$ and $\tau \sigma g_1= \tau g_2 = g_3$. This is how we
can define composition the same in $\C^{A,B}$ as in $\C$. Diagrammatically, this
is like "taking away" the $(Z_2,f_2,g_2)$ object in the joint commutative
diagram for $\sigma$ and $\tau$:
\[
\begin{tikzcd}
A \arrow[dr, "f_1"] \arrow[drr, bend left, "f_2" near end] \arrow[drrr, bend left, "f_3" near end] & & \\
  & Z_1 \arrow[r, "\sigma"] & Z_2 \arrow[r, "\tau"] & Z_3 \\
B \arrow[ur, "g_1"'] \arrow[urr, bend right, "g_2"' near end] \arrow[urrr, bend right, "g_3"' near end] & &
\end{tikzcd}
\hspace{1in}
\begin{tikzcd}
A \arrow[dr, "f_1"] \arrow[drr, bend left, "f_3" near end]    &                               \\
                                                     & Z_1 \arrow[r, "\tau\sigma"] & Z_3 \\
B \arrow[ur, "g_1"'] \arrow[urr, bend right, "g_3"' near end] &
\end{tikzcd}
\]
\end{solution}

\begin{solution}
\def \C {\mathsf{C}}
\def \Call {\C^{\alpha,\beta}}
\def \Cobj {\C^{A,B}}

Let $\C$ be a category. Fix two morphisms $\alpha\in\Hom_\C(C,A)$ and
$\beta\in\Hom_\C(C,B)$ with the same source $C$, and where $A,B,C\in\Obj(\C)$.
We wish to formalize the \textit{fibered} version of $\C^{A,B}$: $\Call$, where
instead of specifying specific objects in $\C$ we use morphisms $\alpha$ and
$\beta$ directly.

The objects in $\Call$ are triples $(Z,f,g)$ where $Z\in\Obj(\C)$,
$f\in\Hom_\C(A,Z)$, and $g\in\Hom_\C(B,Z)$ such that $f\alpha = g\beta$;
intuitively, starting with object $\C$ we can use $\alpha$ and $\beta$ to map to
objects $A$ and $B$, respectively, and the objects in $\Call$ specify a fourth
object $Z$ and morphisms $f:Z\from A$ and $g:Z\from B$ that both map to $Z$.

Morphisms in $\Call$ between objects $(Z_1,f_1,g_1)$ and $(Z_2,f_2,g_2)$ are
morphisms $\sigma\in\Hom_\C(Z_1,Z_2)$ such that everything commutes: $\sigma f_1
\alpha = f_2 \alpha$ and $\sigma g_1 \beta = g_2 \beta$. In short, we diverge to
$A$ and $B$ from $C$, then simultaneously converge to $Z_1$ and $Z_2$ in such a
way that we can continue to $Z_2$ from $Z_1$ mapping with $\sigma$.

\end{solution}



\subsection{Morphisms}


% Problem 4.1
\begin{problem}

$\rhd$ Composition is defined for \textit{two} morphisms. If more than two
morphisms are given, e.g.,
%
\[ A \xrightarrow{f} B \xrightarrow{g} C \xrightarrow{h} D \xrightarrow{i} E \]
%
then one may compose them in several ways, for example,
%
\[ (ih)(gf),\,\,\,\,(i(hg))f,\,\,\,\,i((hg)f),\,\,\,\,\text{etc.}\]
%
so that at every step one is only composing two morphisms. Prove that the result
of any such nested composition is independent of the placement of the
parentheses.
\end{problem}

\begin{solution}
\def \C {\mathsf{C}}
For three morphisms $f,g,h$ in a category $\C$:
%
\[ A \xrightarrow{f} B \xrightarrow{g} C \xrightarrow{h} D \]
%
we have that $(hg)f = h(gf)$ due to $\C$ being a category. Now, fix $n\geq 4$
and suppose that all parenthesizations of $n-1$ morphisms are equivalent.
Imagine that $f_1, \dots, f_n$ are morphisms in a category $\C$:
%
\[ Z_1 \xrightarrow{f_1} Z_2 \xrightarrow{f_2} \cdots Z_n \xrightarrow{f_n}
Z_{n+1} \]
%
Suppose that some parenthesization of $f_n, f_{n-1}, \dots, f_1$ is $f$ and
furthermore that $f=hg$, where $h$ is some parenthesization of $f_n, \dots,
f_{i+1}$, and $g$ is some parenthesization of $f_i, \dots, f_1$, where $1\leq
i\leq n$. Since $h$ and $g$ are parenthesizations of $n-i$ and $i$ morphisms,
respectively, they can be written in the following forms:
%
\[ h = ((\cdots((f_nf_{n-1})f_{n-2})\cdots)f_{i+1}) \]
\[ g = (f_i(f_{i-1}(\cdots(f_2f1)\cdots))) = f_ig' \]
%
in hence $f=hg=h(f_ig')=(hf_i)g'$. Inductively, we can ``pop'' morphisms off the
left hand side of $g'$ and add them to the right hand side of $h$, resulting in
the canonical form:
%
\[ f = ((\cdots((f_nf_{n-1})f_{n-2})\cdots)f_1) \]
\end{solution}


% Problem 4.2
\begin{problem}
$\rhd$ In Example 3.3 we have seen how to construct a category from a set
endowed with a relation, provided this latter is reflexive and transitive. For
what types of relations is the corresponding category a groupoid (c.f. Example
4.6)? [\S 4.1]
\end{problem}

\begin{solution}
\def \C {\mathsf{C}}
Recall that, in order to construct a category from a set $S$ endowed with a
reflexive and transitive relation $\sim$, we take $\Obj(\C) = S$ and set
$\Hom_\C(x,y) = {(x,y)}$ if $x\sim y$, and $\emptyset$ otherwise. This way every
object $x$ has a unique identity morphism $\Hom_\C(x,x)=(x,x)$ since $\sim$ is
reflexive.  Composition of morphisms can be defined since $\sim$ is transitive:
for morphisms $(x,y)$ and $(y,z)$, their composition is $(y,z)(x,y) = (z,x)$.

Now, a groupoid is a category in which every morphism is an isomorphism. We will
show that we can take a pair $(S,\sim)$ where $S$ is a set and $\sim$ is an
\textit{equivalence relation} in order to create a groupoid $\C$. Since $\sim$
is reflexive and transitive, we have seen above how to construct a category
$\C$. In order to show that every morphism is an isomorphism, let
$(x,y)\in\Hom_\C(x,y)$. Hence $x\sim y \implies y\sim x$, since $\sim$ is an
equivalence relation. Now, $(x,y)(y,x) = (y,y)$ and $(y,x),(x,y) = (x,x)$ due to
the morphism composition rule above; hence $(x,y)$ is an isomorphism, so $\C$ is
a groupoid.
\end{solution}


% Problem 4.3
\begin{problem}
\def \C {\mathsf{C}}
Let $A$, $B$ be objects of a category $\C$, and let $f \in \Hom_\C(A, B)$ be a
morphism. 
\begin{itemize}
\item Prove that if $f$ has a right-inverse, then $f$ is an epimorphism.
\item Show that the converse does not hold, by giving an explicit example of a
category and an epimorphism without a right-inverse.
\end{itemize}
\end{problem}

\begin{solution}
\def \C {\mathsf{C}}

Let $A,B,\C,$ and $f$ be as above.
\begin{itemize}
\item Suppose that $f$ has a right-inverse $g:B\to A$ so that $f\circ g: B\to B
= \id_B$. Let $Z\in\Obj(\C)$ and $\beta',\beta'':A\to Z$, and suppose that
$\beta'\circ f=\beta''\circ f$. Then we apply $g$ to both sides to get
$\beta'\circ(f\circ g) = \beta''\circ(f\circ g) \implies \beta'\circ\id_B =
\beta''\circ \id_B$ since $fg=\id_B$, which in turn implies that
$\beta'=\beta''$ since $\id_B$ is the identity.

\item Let $\C$ be such that $\Obj(\C) = \mathbf{Z}$, $\Hom_\C(a,b)=\{(a,b)\}$ if
$a\leq b$ and $\emptyset$ otherwise, and for any objects $a,b,c$ and morphisms
$f:a\to b$ and $g:b\to c$, define $g\circ f$ = $\{(c,a)\}$. Then every morphism
$f\in\Hom_\C(a,b)$ is an epimorphism; this is given in the text. However, if
$f:a\to b = (a,b)$ for $a\neq b$ (hence $a\leq b$,) we have that
$\Hom_\C(b,a)=\emptyset$; so $f$ in general. This implies that epimorphisms do
not in general have right inverses.
\end{itemize}
\end{solution}


% Problem 4.4
\begin{problem}
\def \C {\mathsf{C}}
\def \mono {\mathsf{mono}}
\def \nonmono {\mathsf{nonmono}}

Prove that the composition of two monomorphisms is a monomorphism. Deduce that
one can define a subcategory $\C_\mono$ of a category $\C$ by taking the same
objects as in $\C$ and defining $\Hom_{\C_\mono}(A, B)$ to be the subset of
$\Hom_\C(A, B)$ consisting of monomorphisms, for all objects $A, B$. (Cf.
Exercise 3.8; of course, in general $\C_\mono$ is not full in $C$.) Do the same
for epimorphisms. Can you define a subcategory $\C_\nonmono$ of $\C$ by
restricting to morphisms that are not monomorphisms?
\end{problem}
\begin{solution}
\def \C {\mathsf{C}}
\def \mono {\mathsf{mono}}
\def \nonmono {\mathsf{nonmono}}
\def \epi {\mathsf{epi}}
\def \nonepi {\mathsf{nonepi}}

Let $\C$ be a category with $A,B,C\in \Obj(\C)$, and let $f:A\to B$ and $g:B\to
C$ be monomorphisms. Let $Z\in\Obj(\C)$ and $\alpha',\alpha'':Z\to A$. Suppose
$gf\alpha'=gf\alpha''$. Since $g$ is a mono, $f\alpha'=f\alpha''$. Since $f$ is
a mono, $\alpha'=\alpha''$. Therefore $(gf)\alpha'=(gf)\alpha'' \implies
\alpha'=\alpha''$, so $gf$ is a mono.

This means that we can take the category $\C_\mono$ as detailed in the question.
Since identities are isomorphisms, they are also monomorphisms, so we still have
identities. We just proved that the composition of monomorphisms is a
monomorphism, so the composition of any two appropriate monomorphisms in
$\C_\mono$ between, say $A$ and $B$, and $B$ and $C$, respectively, will also be
a monomorphism hence in $\Hom_{\C_\mono}(A,C)$, so composition ``works'' in
$\C_\mono$.

The $\C_\nonmono$ as described above is not a category since it doesn't have any
identities (since all identities are monomorphisms.)

Now, fix $f:A\to B$ and $g:B\to C$ to be epimorphisms. Let $Z\in\Obj(\C)$ and
$\beta',\beta'':C\to Z$. Suppose $\beta'gf=\beta''gf$. Since $f$ is an epi,
$\beta'g=\beta''g$. Since $g$ is an epi, $\beta'=\beta''$. Hence $gf$ is an epi
as above.

By the same reasoning as above we deduce that $\C_\epi$ is a category and
$\C_\nonepi$ is not a category.
\end{solution}


% Problem 4.5
\begin{problem}
\def \MSet {\mathsf{MSet}}
Give a concrete description of monomorphisms and epimorphisms in the category
$\MSet$ you constructed in Exercise 3.9. (Your answer will depend on the notion
of morphism you defined in that exercise!)
\end{problem}
\begin{solution}
\def \MSet {\mathsf{MSet}}
\def \Set {\mathsf{Set}}
Recall that, for two multisets $\hat{S}=(S,\sim),\hat{T}=(T,\approx)$ (where
$S,T$ are sets and $\sim,\approx$ are equivalence relations on $S$ and $T$,
respectively,) we defined a morphism $f:\hat{S}\to\hat{T}$ in $\MSet$ to be a
normal set-function except with the extra condition that for any $s,s'\in S$, we
require that $f$ preserves equivalence, so if $s\sim s'$ then $f(s)\approx
f(s')$.

The notions of monomorphism and epimorphism transfer over as follows.

\begin{enumerate}
\item A multiset-function $f:\hat{S}\to\hat{T}$ is a monomorphism iff for all
$s_1,s_2\in S$, if $f(s_1)\approx f(s_2)$ then $s_1\sim s_2$.
\item A multiset-function $f:\hat{S}\to\hat{T}$ is an epimorphism iff for all
$t\in T$, there is an $s\in S$ such that $f(s)\approx t$.
\end{enumerate}

(Since $\sim$ and $\approx$ are equivalence relations and since these
definitions are analogous to monos and epis in $\Set$, the proof that these
really are monos and epis is analogous.)
\end{solution}



\subsection{Universal Properties}


% Problem 5.1
\begin{problem}
\def \C {\mathsf{C}}
\def \Cop {{\mathsf{C}^{op}}}
Prove that a final object in a category $\C$ is initial in the opposite category
$\Cop$ (cf. Exercise 3.1).
\end{problem}
\begin{solution}
\def \C {\mathsf{C}}
\def \Cop {{\mathsf{C}^{op}}}
Let $\C$ be a category and suppose that $A\in\Obj(\C)$ is final in $\C$. Then
$\Hom_\C(Z,A)=\Hom_\Cop(A,Z)$ is a singleton for all $Z\in\Obj(\C)$, so $A$ is
initial in $\Cop$.
\end{solution}


% Problem 5.2
\begin{problem}
\def \Set {\mathsf{Set}}
$\rhd$ Prove that $\emptyset$ is the unique initial object in $\Set$. [\S 5.1].
\end{problem}
\begin{solution}
\def \Set {\mathsf{Set}}
Suppose there is another set $I$ which is initial in $\Set$. Then
$\emptyset\simeq I$, so $\abs{\emptyset} = 0 = \abs{I}$. But then vacuously we
get that $\emptyset = I$ (since all the elements in $\emptyset$ are in $I$ and
vice versa,) so $\emptyset$ is the unique initial object in $\Set$.
\end{solution}


% Problem 5.3
\begin{problem}
$\rhd$ Prove that final objects are unique up to isomorphism. [\S 5.1]
\end{problem}
\begin{solution}
\def \C {\mathsf{C}}
Let $\C$ be a category and $F_1,F_2$ be two final objects in $\C$. Then there
are unique morphisms $f:F_1\to F_2$ and $g:F_2\to F_1$. Since there are only one
of each identities $1_{F_1}$ and $1_{F_2}$, then necessarily $gf = 1_{F_2}$ and
$fg = 1_{F_1}$, hence $f$ is an isomorphism.
\end{solution}


% Problem 5.4
\begin{problem}
What are initial and final objects in the category of `pointed sets' (Example
3.8)? Are they unique?
\end{problem}
\begin{solution}
\def \Set {\mathsf{Set}}
\def \SetS {\mathsf{{Set^*}}}
Recall that $\SetS$ is the set of pairs $(S,s)$ where $S$ is a set and $s\in S$.
We claim that objects $(\{s\},s)$, i.e. pointed singleton sets, are the initial
and final objects in $\SetS$. Note that there can be no "empty function" between
pointed sets, since each set has to have a point. Suppose $(T,t)\in\Obj(\SetS)$.
Then there is only one function $f:S\to T$ such that $f(s)=t$: the function
$f=\{(s,t)\}$. There is also only one function $f:T\to S$, namely the function
that maps each element $t$ in $T$ to $s$. Hence singleton pointed sets are
initial and final.

Furthermore, clearly morphisms between pointed sets $(S,s)$ and $(T,t)$ such
that $\abs{S},\abs{T}\geq 2$, there are more than one function $f:S\to T$ and
$g:T\to S$: we could take $f(s)=f(s')=t$, or $f(s)=t,f(s')=t'$.

They are not unique; any singleton pointed set is initial and final.
\end{solution}


% Problem 5.5
\begin{problem}
Consider the category corresponding to endowing (as in Example 3.3) the set
$\mathbf{Z}^+$ of positive integers with the divisibility relation. Thus there
is exactly one morphism $d\to m$ in this category if and only if $d$ divides $m$
without remainder; there is no morphism between $d$ and $m$ otherwise. Show that
this category has products and coproducts. What are their `conventional' names?
[\S VII.5.1]
\end{problem}
\begin{solution}
\def \Div {\mathsf{Div}}
Let $\Div$ be the above category. Let $m,n\in\Obj(\Div)$. We claim that
$\gcd(m,n)$ corresponds to a final object (namely $(\gcd(m,n),m,n)$) in
$\Div_{m,n}$. Note that for any $z\in\Obj(\Div)$ such that $z\mid m$ and $z\mid
n$, $z\mid\gcd(m,n)$ (by definition of $\gcd$;) hence
$\Hom_{\Div_{m,n}}((z,m,n),(\gcd(m,n),m,n))$ is non-empty. Furthermore, since
there can only be at most 1 morphism between any two objects in $\Div$,
$(\gcd(m,n),m,n)$ is final. The conventional name for this is the `greatest
common divisor.'

The coproducts in $\Div$ are the `least common multiple'. For any
$z\in\mathbf{Z}^+$, if $m\mid z$ and $n\mid z$, then $\lcm(n,m)\mid z$. Hence
$((\lcm(m,n),m,n),(z,m,n))$ is the unique morphism from $(\lcm(m,n),m,n)$ in
$\Div^{m,n}$, so $(\lcm(m,n),m,n)$ is initial.
\end{solution}


% Problem 5.5
\begin{problem}
What are the final objects in the category considered in \S5.3? [\S5.3]
\end{problem}
\begin{solution}
\end{solution}


% Problem 5.6
\begin{problem}
Redo Exercise 2.9, this time using Proposition 5.4.
\end{problem}
\begin{solution}
\def \Set {\mathsf{Set}}
Suppose $A,B,A',B'$ are sets with $A\cap B=\emptyset$, $A'\cap B'=\emptyset$,
$A\cong A'$, and $B\cong B'$. We will show that there are two isomorphic
disjoint unions corresponding to $A\cup B$ and $A'\cup B'$.

First, take $i_A:A\to A\cup B, i_A(a) = a$ for all $a\in A$ and analogous for
$B$. Then if $Z$ is a set with morphisms $f_A:A\to Z$ and $f_B:B\to Z$, we can
take $\sigma:A\amalg B=A\cup B\to Z, \sigma(x)$ to be $f_A(x)$ if $x\in A$ and
$f_B(x)$ otherwise. This is analogous to the proof for disjoint union being a
coproduct, hence $A\amalg B=A\cup B$ is a disjoint union.

Second, since $A\cong A'$ and $B\cong B'$, let $f:A\to A'$ and $g:B\to B'$ be
isomorphisms. We can take $i_{A'}: A\to A'\cup B',i_{A'}(a) = f(a)$ for all
$a\in A$ and similar for $i_{B'}$. Then if $Z$ is a set with morphisms $f_A:A\to
Z$ and $f_B:B\to Z$, we can take $\sigma:A'\amalg B'=A'\cup B'\to Z, \sigma(x)$
to be $f_A\circ f^{-1}$ if $x\in A'$ and $f_B\circ g^{-1}$ otherwise (which
works since $A'\cap B'=\emptyset$.) Hence $A'\amalg B'=A'\cup B'$ is a disjoint
union.

By Proposition 5.4, since both $A\amalg B$ and $A'\amalg B'$ are initial objects
in some auxiliary category of $\Set$, they are isomorphic, as required.
\end{solution}


% Problem 5.7
\begin{problem}
\def \C {\mathsf{C}}
Show that in every category $\C$ the products $A\times B$ and $B\times A$ are
isomorphic, if they exist. (Hint: Observe that they both satisfy the universal
property for the product of $A$ and $B$; then use Proposition 5.4.)
\end{problem}
\begin{solution}
\def \C {\mathsf{C}}
Let $\C$ be a category with products $A\times B$ and $B\times A$. First,
consider how $f:A\times B\to B\times A, f(a,b) = (b,a)$ is an isomorphism
between $A\times B$ and $B\times A$ (with inverse $f^{-1}(b,a) = (a,b)$.) Since
$B\times A$ is a product in $\C$, for each object $Z\in\Obj(\C)$ with morphisms
$f_B:B\to Z$ and $f_A:A\to Z$, there is a unique morphism $\tau:Z\to B\times A$
such that everything commutes. However, using $f$ we can construct a unique
morphism $\sigma:Z\to A\times B$ in terms of $\tau$ by taking $\sigma =
f^{-1}\circ\tau$. Hence $B\times A$ is a product for $A\times B$ as well, i.e.
$B\times A$ is a final object in some auxiliary category.

Hence, by Proposition 5.4, $A\times B$ and $B\times A$ are isomorphic.
\end{solution}


% Problem 5.8
\begin{problem}
\def \C {\mathsf{C}}
Let $\C$ be a category with products. Find a reasonable candidate for the
universal property that the product $A\times B\times C$ of three objects of $\C$
ought to satisfy, and prove that both $(A\times B)\times C$ and $A\times
(B\times C)$ satisfy this universal property. Deduce that $(A\times B)\times C$
and $A\times (B\times C)$ are necessarily isomorphic.
\end{problem}
\begin{solution}
\def \C {\mathsf{C}}
\def \tp {A\times B\times C}
Let $\C$ be a category with products, and let $A,B,C\in\Obj(\C)$. The
three-product is an object $\tp\in\Obj(\C)$ with morphisms $\pi_A:\tp\to A$,
$\pi_B:\tp\to B$, and $\pi_C:\tp\to C$ such that for all $Z\in\Obj(\C)$ with
morphisms $f_A:Z\to A,f_B:Z\to B,f_C:Z\to C$, there is a unique morphism
$\sigma:Z\to\tp$ such that the following diagram commutes:
\[\begin{tikzcd}
&&A \\
Z\arrow[r,"\sigma"]\arrow[urr,bend left,"f_A"]\arrow[drr,bend right,"f_B"]\arrow[ddrrr,bend right,"f_C"]&
      \tp\arrow[ur,"\pi_A"]\arrow[dr,"\pi_B"]\arrow[ddrr,bend left,"\pi_C"] \\
&&B \\
&&&C
\end{tikzcd}\]

First, we will show that $(A\times B)\times C$ is a three-product. Since
$A\times B$ and $Z\times C$ are products, there are a unique morphisms
$\tau:A\times B\to Z$ and $\upsilon:Z\times C\to Z$ for every object $Z$.  We
can use these two morphisms to build $\sigma:\tp\to Z$ for any object $Z$ as
follows: $\sigma:(A\times B)\times C\to Z, \sigma(a,b,c)=\upsilon(\tau(a,b),c)$.
Since $\upsilon$ and $\tau$ are well-defined and unique, $\sigma$ is
well-defined and unique. Hence $(A\times B)\times C$ is a three-product.

Now, consider $A\times(B\times C)$. Similarly, this corresponds to unique
morphisms $\tau:A\times Z\to Z$ and $\upsilon:B\times C\to Z$ from which we can
construct $\sigma:A\times(B\times C)\to Z,\sigma(a,b,c)=\tau(a,\upsilon(b,c))$.
By the same logic as above, $A\times(B\times C)$ is a three product.

Thus by Proposition 5.4, $(A\times B)\times C$ and $A\times(B\times C)$ are
isomorphic.
\end{solution}


% Problem 5.10
\begin{problem}
\def \Set {\mathsf{Set}}
Push the envelope a little further still, and define products and coproducts
for families (i.e., indexed sets) of objects of a category.

Do these exist in $\Set$?

It is common to denote the product $A\times\cdots\times A$ ($n$ times) by $A^n$.
\end{problem}
\begin{solution}
\def \C {\mathsf{C}}
\def \Set {\mathsf{Set}}
Let $\C$ be a category and $I$ be a set. Consider $\{A_i\}_{i\in I}$ with each
$A_i\in\Obj(\C)$. An \textit{infinitary product} $\prod_{i\in I}
A_i\in\Obj(\C)$ with morphisms $\{\pi_{A_i}\}_{i\in I}$ must satisfy the
universal property that, for all $Z\in\Obj(\C)$ and morphisms $\{f_{A_i}\}_{i\in
I}$, there must be a unique $\sigma:Z\to\prod_{i\in I}A_i$ such that
$\sigma\pi_{A_i}=f_{A_i}$ for all $i\in I$.

These should exist in $\Set$ as long as we have the axiom of choice.
\end{solution}


% Problem 5.11
\begin{problem}
\newcommand{\quot}[2]{#1/\!\!#2\,\,}
\newcommand{\quotntws}[2]{#1/\!\!#2}
Let $A$, resp. $B$, be a set, endowed with an equivalence relation $\sim_A$,
resp. $\sim_B$.
Define a relation $\sim$ on $A\times B$ by setting
\[ (a_1, b_1) \sim (a_2, b_2) \iff a_1 \sim_A a_2 \text{ and } b_1 \sim_B b_2. \]
(This is immediately seen to be an equivalence relation.)
\begin{itemize}
\item Use the universal property for quotients (\S5.3) to establish that there are
functions
\[ \quot{(A\times B)}{\sim} \to \quotntws{A}{\sim_A},
   \quot{(A\times B)}{\sim} \to \quotntws{B}{\sim_B}. \]
\item Prove that $\quotntws{(A\times B)}{\sim}$, with these two functions,
satisfies the universal property for the product of $\quotntws{A}{\sim_A}$ and
$\quotntws{B}{\sim_B}$.
\item Conclude (without further work) that $\quot{(A\times
B)}{\sim}\cong(\quotntws{A}{\sim_A})\times(\quotntws{B}{\sim_B}).$
\end{itemize}
\end{problem}
\begin{solution}
\newcommand{\quot}[2]{#1/\!\!#2\,\,}
\newcommand{\quotntws}[2]{#1/\!\!#2}
Let $A,B,\sim,\sim_A,\sim_B$ be as above. Let $\pi_A:A\times B\to A$ and
$\pi_B:A\times B\to B$ be the product canonical projections for $A$ and $B$. Let
$\pi^Z_\sim:Z\to\quotntws{Z}{\sim}$ be the canonical quotient mapping for all
objects $Z$ and equivalence relations $\sim$. Then we can apply the universal
property for quotients twice to get the required two functions:
\[\begin{tikzcd}[column sep=tiny]
\quotntws{(A\times B)}{\sim} \arrow[rr,"\quotuniv{\pi^A_{\sim_A}\circ\pi_A}"] && \quotntws{A}{\sim_A} \\
&A\times B \arrow[ul,"\pi^{A\times B}_\sim"] \arrow[ur,swap, "\pi^A_{\sim_A}\circ\pi_A"]&
\end{tikzcd}\]

\[\begin{tikzcd}[column sep=tiny]
\quotntws{(A\times B)}{\sim} \arrow[rr,"\quotuniv{\pi^B_{\sim_B}\circ\pi_B}"] && \quotntws{A}{\sim_A} \\
&A\times B \arrow[ul,"\pi^{A\times B}_\sim"] \arrow[ur,swap, "\pi^B_{\sim_B}\circ\pi_B"]&
\end{tikzcd}\]

\def \quotA {\quotntws{A}{\sim_A}}
\def \quotB {\quotntws{B}{\sim_B}}

Now, we wish to show that $\quotntws{(A\times B)}{\sim}$ satisfies the universal
property for the product of $\quotA$ and $\quotB$. Rename the two functions
proved above to be $*^A$ and $*^B$. Let $Z$ be a set with morphisms
$f_A:Z\to\quotA$ and $f_B:Z\to\quotB$. We wish to construct a function $\sigma$
so that the following diagram commutes:
%
\[\begin{tikzcd}
&& \quotntws{A}{\sim_A} \\
Z\arrow[r,"\sigma"]\arrow[urr,bend left,"f_A"]\arrow[drr,bend right,swap,"f_B"]
   &\quotntws{(A\times B)}{\sim}\arrow[ur,"*^A"]\arrow[dr,swap,"*^B"] \\
&& \quotntws{B}{\sim_B}
\end{tikzcd}\]
%
\def \quotAB {\quotA\times\quotB}
First, define $f:Z\to\quotAB$ to be $f(z)=(f_A(z),f_B(z))$. Next, we observe
that we can use the quotient universal property with $\quotA$ to get a map
$\quotuniv{1_A}:\quotA\,\,\to A$ and likewise for $\quotuniv{1_B}:\quotB\,\,\to
B$. Define $\quotuniv{1_{A\times B}}:\quotAB\,\,\to A\times B$ to be
$\quotuniv{1_{A\times B}}([a]_{\sim_A},[b]_{\sim_B}) =
(\quotuniv{1_A}([a]_{\sim_A}), \quotuniv{1_B}([b]_{\sim_B}))$. Finally, we can
take $\sigma=\pi^{A\times B}_\sim \circ \quotuniv{1_{A\times B}} \circ
f:Z\to\quotntws{(A\times B)}{\sim}$ to satisfy the universal property for
product of $\quotA$ and $\quotB$ (it is uniquely determined by its respective
pieces.)

Therefore, by Proposition 5.4, $\quot{(A\times B)}{\sim}\cong\quotA \times
\quotB$.
\end{solution}


% Problem 5.12
\begin{problem}
\def \C {\mathsf{C}}
\def \Set {\mathsf{Set}}
Define the notions of fibered products and fibered coproducts, as terminal
objects of the categories $\C^{\alpha,\beta}, \C_{\alpha,\beta}$ considered in
Example 3.10 (cf. also Exercise 3.11), by stating carefully the corresponding
universal properties.

As it happens, $\Set$ has both fibered products and coproducts. Define these
objects `concretely', in terms of naive yet theory. [II.2.9, III.6.10, III.6.11]
\end{problem}
\begin{solution}
I'm not really sure how to phrase this yet. I think I'll come back to it later
when we've dealt with fibered categories a little more.
\end{solution}
\end{document}
